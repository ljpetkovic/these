\section{Regards croisés sur la formation des concepts}
Étant donné l'importance des travaux de Charcot et ses contributions dans le domaine de la neurologie et au-delà, nous souhaitons explorer la notion de la circulation des savoirs au prisme du numérique à travers son impact. Avant d'aborder la question d'opérationnalisation de son impact, nous tenons d'abord à décortiquer les mécanismes à l'origine des circulations des savoirs à grande échelle, ainsi que de définir la notion d'un \og{}concept\fg{} pouvant véhiculer les informations importantes concernant les circulations en question. 
\subsection{Histoire et épistémologie}
\label{concept}

Afin de pouvoir analyser les concepts médicaux liés à Charcot, il est important de déterminer de quelle manière un mot ou un groupe de mots devient un concept général ou scientifique. Les termes \textit{idée}, \textit{concept}, \textit{terme}, \textit{mot} et \textit{mot-clé} figurent parmi des notions fondamentales dans les disciplines aussi théoriques (linguistique générale, épistémologie ou philosophie) que numériques ou celles ayant un aspect appliqué, comme p. ex. \textsc{TAL} et \textsc{HN}. 
Malgré leur présence répandue dans les domaines cités, ainsi que leur utilisation devenue quasi banale dans le langage courant, ces notions demeurent sans définition fixe et universellement acceptée en raison de la disparité des contextes dans lesquels elles sont utilisées. En plus, elles sont interdépendantes et la frontière entre eux est floue. 

Concernant la notion du concept, quelques remarques philosophiques de \citeauthor{Lecourt1999} (\citeyear{Lecourt1999}, p.~261-263) méritent d'être soulignées ici. Premièrement, l'invention de l'entité du concept remonte à l'ère d'Aristote, qui l'a caractérisé comme une abstraction, un mode de connaissance médiat et général, et comme mode de classification entre le genre et l'espèce (\textit{intension} et \textit{extension}, respectivement). L'intension du concept de chat est sa définition : \og{}animal à quatre pattes de la famille des félins\fg{}, tandis que son extension est un chat concret : le chat tigré, mon chat etc. Deuxièmement, un concept décrit un sujet, il est définissable et représente un résultat de l'abstraction du donné\footnote{Le concept de \og{}donné\fg{} est utilisé en philosophie pour désigner \og{}ce qui est immédiatement présent à l'esprit avant que celui-ci n'y applique ses procédés d'élaboration\fg{}, \url{http://stella.atilf.fr/}.} empirique qui forme une de ses extensions. Cette notion n'est pas à confondre avec celle de l'\textit{idée}, qui représente elle-même l'objet de connaissance et la condition même du concept, distinction faite de manière systématique chez Kant. Finalement, au-delà des définitions du concept présentées ci-dessus du point de vue phénoménologique à travers l'intension et l'extension, la notion du concept peut également être comprise comme un élément d'un jugement qui peut être une loi scientifique. En d'autres mots, la conception d'un concept inclut non seulement les descriptions d'un sujet en utilisant les prédicats à une place (a.), mais s'étend aussi aux relations \textit{n}-aires (b.) ou même à celles entre des concepts plus abstraits qui impliquent des propriétés allant au-delà des simples prédicats (c.). Cette théorie plus \og{}inférentielle\fg{} est à l'origine des concepts scientifiques, dont l'illustration nous retrouvons dans les exemples suivants :
\begin{itemize}
	\item[\quad (a.)] \og{}le chat est roux\fg{} : \textit{le chat} est un sujet (concept) et \textit{être roux} est un prédicat ;
	\item[\quad (b.)] \og{}le chat voit un chien\fg{} : le sujet \textit{le chat} forme une relation binaire avec un objet \textit{un chien} à l'aide du prédicat \textit{voir} ;
	\item[\quad (c.)] \og{}Dans un \textit{triangle rectangle}, le \textit{carré} de la \textit{longueur} de l'\textit{hypoténuse} est \textit{égal} à la \textit{somme} des \textit{carrés} des \textit{longueurs} des deux \textit{côtés} de l'\textit{angle droit}\fg{} : les concepts mathématiques sont typographiés en italique.
\end{itemize}
\medskip

Nous juxtaposons ce point de vue aux réflexions sociohistoriques de \citeauthor{stengers1987d} (\citeyear{stengers1987d}) qui rendent compte des particularités des concepts scientifiques. D'après elle, l'attribut \textit{scientifique} est associé à leur objectivité et leur puissance explicative, or il n'implique pour autant pas une neutralité d'avis qui est considérée néfaste pour les recherches scientifiques et en même temps fictive. L'autrice renforce cette idée en prétendant que le concept scientifique est forcément controversé, puisqu'il est sujet aux discussions, aux polémiques et aux consensus, ce qui impose une prise de position. Le concept scientifique a des rôles particuliers dans les opérations régissant un champ scientifique, notamment sa singularité, son pouvoir d'extension et d'organisation effective des phénomènes, en s'opposant ainsi à la simple présentation des idées de la part de son$\cdot$sa émetteur$\cdot$trice, tout en comprenant un aspect polémique \citep[pp.~10-11]{stengers1987d}. 

À ces traits s'ajoute celui que la même autrice appelle \og{}la propagation épidémique\fg{} (p.~16), où les domaines \og{}infectés\fg{} par un concept scientifique peuvent être autonomes et devenir une source de nouvelle propagation. Cela est illustré sur l'exemple du concept \og programme \fg{} en biologie (matériel génétique et sa fonction) qui a migré vers le domaine de l'informatique (opération d'un ordinateur). Les concepts sont donc capables de voyager d'une science à l'autre, ce qui a inspiré la métaphore des \og{}concepts nomades\fg{}, marqués par leur circulation spatio-temporelle et linguistique. Outre la nature itinérante des concepts scientifiques qui contribue à l'interdisciplinarité et à la production des savoirs nouveaux, \citeauthor{stengers1987d} (\citeyear{stengers1987d}, pp.~21-23) se réfère aux opérations de la \og capture \fg{} de la scientificité par ces concepts et du \og durcissement \fg{} conséquent des sciences. À savoir, certains concepts atteignent le degré de maturité après s'être avéré être adéquats et pertinents dans les démarches scientifiques dont ils \og{}capturent\fg{} la scientificité, permettant ainsi que le statut des sciences se solidifie ou \og durcisse \fg{}. La capture implique la définition, mais aussi la redéfinition d'une notion par les spécialistes d'une science. Les points de vue de \citeauthor{stengers1987d} (\citeyear{stengers1987d}) relèvent de la théorie constructiviste du savoir scientifique, selon laquelle la science est une \og{}construction\fg{} collective issue du contexte socio-historique (p. ex. interaction entre les scientifiques, les institutions etc.), et non pas d'une accumulation neutre et objective de faits.

Cette approche est complémentaire à l'histoire des concepts (allem. \textit{Begriffsgeschichte}), dans laquelle les significations des concepts en général sont considérées d'être les dérivés d'un contexte sociopolitique. Plus précisément, cette transformation d'un ou plusieurs mots en un concept survient lorsque cette construction linguistique comprend toute la gamme des significations dérivées d'un tel contexte \citep[p.~19]{koselleck2011introduction}. À titre d'exemple, le concept d'un \textit{état} ne peut être interprété qu'à travers ses différents constituants, dont \textit{souveraineté territoriale, législation, fiscalité}, parmi maints d'autres. L'histoire des concepts concerne principalement les manifestations de conflits sociopolitiques particuliers qui doivent être compris dans leur contexte approprié, où p. ex. les mots comme \textit{liberté} ou \textit{démocratie} portent la connotation polémique dont le sens ne peut être précisé qu'à travers leurs antithèses (\textit{esclavage} et \textit{dictature}, respectivement). Les concepts sont donc les concentrations par défaut ambiguës d'une multitude de contenus sémantiques, uniquement interprétables et indéfinissables, par contraste avec des significations des mots qui peuvent être définies de manière exacte \citep[p. 20]{koselleck2011introduction}. De plus, les concepts comme \textit{histoire} ou \textit{progrès} sont caractérisés comme \og{}collectifs singuliers\fg{} qui marquent un passage du domain concret d'un individu (plusieurs \textit{histoires} et \textit{progrès} individuels) au domain abstrait et général du collectif social (une \textit{histoire} ou un \textit{progrès} général ou collectif). Ce phénomène linguistique, ainsi que la création des concepts comme \textit{industrie, usine, classe moyenne} etc., reflète un changement de paradigme dans l'organisation sociale survenu lors des révolutions politiques et industrielles \citep[p. 1]{hobsbawm2010age}. La période charnière concernée par ce phénomène est nommée \textit{Sattelzeit} (allem. \og{}époque de selle\fg{}), entre 1750 et 1850, durant laquelle les concepts historiques deviennent abstraits, singularisés, respatialisés et retemporalisés \citep[pp.~34-35]{koselleck2011introduction}. Cela traduit le lien fort entre l'histoire du langage et l'histoire des idées.

Ces considérations sont appliquables à d'autres \og{}concepts nomades\fg{} en sciences humaines et sociales (ci-après \textsc{SHS}), comme \textit{travail}, \textit{intelligencija}, \textit{Ancien Régime}, \textit{avant-garde}, \textit{Occident} etc. qui font partie du \textit{Dictionnaire des concepts nomades en sciences humaines} \citep{christin2011dictionnaire}. Plusieurs questionnements ont été soulevés par \citeauthor{ghermani2011} (\citeyear{ghermani2011}, p.~117) eu égard de leur émergence, notamment pour déterminer à quel moment un concept devient une entrée dans un dictionnaire des \textsc{SHS} : \og{}\textit{Pourquoi un concept fait-il son entrée dans un dictionnaire ? Au terme de quel processus ? À l'inverse, comment cette percée lexicale est-elle parfois impossible ou refusée ?}\fg{}. Contrairement aux processus de la propagation et de la capture qui permettaient à un concept d'obtenir le statut de scientificité, l'autrice souligne les pratiques scientifiques conduisant aux rétractations et aux masquages de sens des concepts en \textsc{SHS}, p. ex. dans le cas du terme \og{}confession [religieuse]\fg{}, dont le sens varie en fonction de l'historiographie dans laquelle il figure \citep[p.~117]{ghermani2011}. Enfin, \citeauthor{bal2002travelling} (\citeyear{bal2002travelling}, p.~34) va plus loin en excluant la \og diffusion \fg{} et en mettant en avant la \og propagation \fg{} comme le critère discriminatoire de la nature itinérante des concepts. 

Pour résumer la complexité de la définition des concepts du point de vue de leur histoire, nous citons ici \citeauthor{bal2002travelling} (\citeyear{bal2002travelling}, p.~51), selon laquelle les concepts sont :
\begin{itemize}
	\item datés, et donc marqués par une évolution ;
	\item les mots : archaïsmes et néologismes relevant des mécanismes étymologiques qui leur donnent une dimension philosophique ;
	\item syntaxiques au sein d'une langue ;
	\item en évolution constante ;
	\item créés, et non pas donnés \textit{a priori}.
\end{itemize}
\medskip
Concernant plus précisément le concept scientifique, l'épistémologie en esquisse les traits suivants, comme souligné par \citeauthor{rumelhard1986} (\citeyear{rumelhard1986}) et cité dans \citeauthor{astolfi2008chapitre} (\citeyear{astolfi2008chapitre}, p.~25) :
\begin{itemize}
	\item le concept scientifique possède une dénomination et une définition, avec le sens le plus univoque possible, \textit{a contrario} du concept linguistique, en principe équivoque et polysémique ;
	\item fonction opératoire : le concept scientifique est un outil intellectuel, un instrument théorique permettant d'interpréter des phénomènes ;
	\item fonction d'opérateur, caractérisé par son degré de formalisation et par les interconnexions avec les techniques scientifiques ;
	\item une extension, une compréhension, un domaine et des limites de validités en lien étroit avec sa définition fixée ;
	\item le concept scientifique peut être compris comme un n\oe{}ud dans un réseau de relations organisé, au sein duquel il dialogue avec d'autres concepts et théories scientifiques.
\end{itemize}


%\section{Repérage des termes scientifiques dans un corpus numérique}
\subsection{Linguistique computationnelle}
%\label{termes}
\hl{Ajouter une définition du concept du point de vue de TAL}
Si nous nous limitons aux théories abordées jusqu'à maintenant, nous pouvons considérer que les concepts médicaux de Charcot ont eu le rôle des vecteurs de la crise conceptuelle, ce qui représentait une forme de \textit{Sattelzeit} dans le domaine de la médecine. Autrement dit, ces concepts ont été détournés de leurs sens initiaux ayant une apparence formelle neutre (descriptions des pathologies), vers ceux exerçant un certain impact sur la communauté scientifique que nous souhaitons mesurer informatiquement. Néanmoins, l'analyse numérique des concepts n'est pas une tâche triviale non plus, car tous les logiciels ne traitent pas des textes de la même manière. D'après \citeauthor{silberztein2022linguistic} (\citeyear{silberztein2022linguistic}, p.~2), les logiciels comme \textsc{TXM}\footnote{\url{https://txm.gitpages.huma-num.fr/textometrie/}}, Sketch Engine\footnote{\url{https://www.sketchengine.eu/}} ou \textsc{IR}a\textsc{M}u\textsc{T}e\textsc{Q}\footnote{\url{http://www.iramuteq.org/}} traitent les documents comme des \textit{séquences de formes graphiques} (dans notre cas, les séquences \og{}hystérie\fg{} et \og{}arthrite déformante\fg{} seraient composées d'une et de deux formes graphiques, respectivement). Ces formes sont définies comme les séquences contiguës de caractères alphabétiques délimités par des non-lettres ou les délimiteurs, qui peuvent être considérées comme des informations potentiellement  pertinentes pour une étude. D'autres logiciels, comme \textsc{N}oo\textsc{J}\footnote{\url{https://nooj.univ-fcomte.fr/}}, peuvent traiter ces séquences comme les \textit{unités linguistiques atomiques}, quel que soit le nombre de formes graphiques \citep[pp.~2-3]{silberztein2022linguistic}. Ainsi, l'unité linguistique atomique \og{}hystérie\fg{} serait recensée dans un dictionnaire des entrées lexicales simples (\textsc{DELAS}), tandis que \og{}arthrite déformante\fg{} ferait partie du dictionnaire des entrées lexicales composées (\textsc{DELAC})\footnote{Ce principe est repris lors du développement du logiciel Unitex \url{https://unitexgramlab.org/fr}.}.

Afin d'extraire automatiquement les concepts scientifiques, nous les opérationnalisons comme des \textit{termes} scientifiques. On en trouve une analogie proche dans la distinction terminologique relevée par \citeauthor{saussure1915} (\citeyear{saussure1915}, pp.~74-75) entre un \textit{signifié} (p. ex. le concept d'un arbre dans notre système cognitif) et un \textit{signifiant} (mot, parole, pictograme désignant un arbre) qui consitue un \textit{signe} (référent, un arbre réel). Les termes sont des expressions textuelles et unités sémantiques qui désignent des concepts dans un domaine d'expertise spécifique. Par conséquent, la tâche d'extraction des concepts peut donc être formalisée comme un problème d'extraction de la terminologie (angl. \textit{automatic text extraction -- \textsc{ATE}}), dont les enjeux appartiennent au domaine de l'extraction d'information (angl. \textit{information retrieval}), et plus largement, à celui du \textsc{TAL}. L'\textsc{ATE} a pour objectif de faciliter l'identification manuelle des termes à partir de corpus spécifiques à un domaine en fournissant une liste de termes candidats \citep[p.~1]{tran2023recent}. Jusqu'à maintenant, trois grandes méthodes d'extraction de la terminologie ont été recensées dans la littérature : linguistique, statistique et la méthode basée sur les apprentissages machine et profond (angl. \textit{machine learning} et \textit{deep learning}, respectivement).
\hl{À FAIRE}
\begin{itemize}
	\item TermoStat\footnote{\url{https://termostat.ling.umontreal.ca/doc_termostat/doc_termostat.html}} \citep{drouin2003} : termes simples vs. complexes nominaux ; à base de règles ; limite de corpus : 30 Mo + connexion échouée ou phénomène de bottleneck ; extraction des POS
	\item extraction terminologique \texttt{TermSuite}\footnote{\url{https://termsuite.github.io/}} \citep{cram2016terminology} : scalable, \texttt{TreeTagger}
	\item approche linguistique : besoin d'expert du domaine, analyse syntaxique, POS tagging qui a ses limitations, ne peut pas mesurer la pertinence du terme
	\item approche statistique, \textsc{TF-IDF}, \textsc{BM25} \citep{robertson1976relevance}, pas besoin d'expert du domaine, mesure de pertinence : \textit{termhood} et \textit{unithood} \citep[pp.~6-7]{kageura1996} vs. fréquence
	\item approche apprentissage machine / profond, \texttt{keybert} \citep{grootendorst2023}\footnote{\url{https://maartengr.github.io/KeyBERT/}}, \texttt{keyphrase-vectorizers}\footnote{\url{https://pypi.org/project/keyphrase-vectorizers/}}  : \citep{tran2023recent}
	\item Pour nous, concept scientifique est opérationnalisé comme un terme scientifique.
\end{itemize}




%Motasem :
%revenir sur les concepts de l'index
%comment identifier les concepts médicaux dans les textes ? 
%regarder les fréquences des concepts médicaux
%quelles formes ?
%comment les identifier dans un texte ?
%lire la proposition de Motasem sur les anecdotes et celle de Glenn sur les embeddings dynamiques
%\textsc{ATISHS} : outil pour les HN 
%faire les stats à partir de la liste des concepts -- index (cooccurrences, pattern matching etc.) : qu'est-ce qui marche et marche pas ?

\textbf{Comment définir les concepts scientifiques du point de vue du TAL / analyse du corpus ? concepts, termes et mots-clés}



%Le mot \og{}concept\fg{} est un terme générique qui renvoie à un grand nombre de théories provenant de divers domaines de pensée, sans qu'il en existe une qui soit exhaustive et universellement acceptée.



%En revanche, selon les linguistes, un concept a une structure double, constituée du sens linguistique et culturel.
%%dont linguistique (générale, cognitive, psycholinguistique, ethnolinguistique), philosophie, métaphysique ou mathématiques
%Sa couche intérieure est constituée du noyau étymologique sur lequel repose ensuite la couche périphérique qui hérite les éléments formés par la culture, les traditions et les expériences humaines
%%\foreignlanguage{russian}{(Степанов, \citeyear{stepanov2007}}). 
%\footnote{En linguoculturologie, on retrouve le terme \og{}concept linguo-culturel\fg{} qui reflète cette nature double du concept.}. Il peut être exprimé par de différentes éléments du langage, soit : lexèmes, idiomes, collocations, phrases ou textes entiers \citep[p.~5]{nemickiene2011concept}. 

Dans le domaine du traitement automatique des langues (\textsc{TAL}), le terme \og concept \fg{} peut s'apparenter à celui des \og entités nommées \fg{}, comme en témoignent les recherches sur l'extraction automatique de la terminologie biomédicale (\citealp{jolly2024exploring,navarro2023clinical}). Un concept d'un domaine de connaissance peut faire partie d'un thésaurus, liste organisée de termes contrôlés et normalisés, auquel cas le concept est appelé \og descripteur \fg{}. \citep[p.~16]{RENNESSON202015}.

Un exemple de ce phénomène est le terme \textsc{mot}, qui véhicule une réalité particulière appartenant à chaque langue (\citeauthor{mounin1968clefs} \citeyear{mounin1968clefs}, p. 65). 

Nous n'entendons pas le terme \textsc{concept} dans le sens de Saussure,.... signe = concept (signifié) + image acoustique (signifiant)

Même si l'on reprend la description de Saussure qui considère le mot comme \og une image acoustique associé à un concept \fg{}, nous nous heurtons ensuite au problème de la définition du terme \textit{concept}. Le structuralisme linguistique de Bloomfield souligne ce point, en ajoutant que les linguistes ne sont pas outillés pour démêler complètement ce réseau complexe. Ce structuraliste poursuit en disant que le langage peut en effet être perçu comme une abstraction construite à partir de nos connaissances sur celui-ci, mais qu'il faut \og décrire d'abord le fonctionnement de cet instrument de communication \fg{} et expliquer comment nous (dé)construisons les énoncés en tant que locuteurs ou auditeurs (\citeauthor{mounin1968clefs} \citeyear{mounin1968clefs}, pp. 94-95).

\begin{itemize}
	\item ok, et c'est quoi le concept en linguistique (de Saussure) et en analyse du discours
	\item nous différencions des concepts des \og{}figements linguistiques\fg{} \citep{bezancon2023}
\end{itemize}



Dans le souci de différencier ces notions à travers les disciplines citées, nous présentons ci-dessous quelques-uns de leurs traits discriminatoires qui ne prétendent être ni exhaustifs ni limitatifs :

\begin{table}[h]
	\centering
	\begin{tabular}{|l|l|l|l|}
		\hline
		& \multicolumn{1}{c|}{\begin{tabular}[c]{@{}c@{}}Philosophie\\ Épistémologie\end{tabular}} & \multicolumn{1}{c|}{Linguistique} & \multicolumn{1}{c|}{TAL} \\ \hline
		\textsc{Idée}    & objet de connaissance                                                                                          \citep[p.~261]{Lecourt1999} &                                   &                          \\ \hline
		\textsc{Concept} & représentation de l'objet de connaissance \citep[p.~261]{Lecourt1999}                                                                                      \\ \hline 
		\textsc{Signifié} &  \citep[p.~27]{astolfi2008chapitre}                                                                                      \\ \hline
		\textsc{Signifiant} &    mode de représentation des signifiés \citep[p.~27]{astolfi2008chapitre}                              &                          \\ \hline
		\textsc{Terme}   &                                                                                          &                                   &                          \\ \hline
		\textsc{Mot} &                                                                                          &                                   &                          \\ \hline
		\textsc{Mot-clé} &                                                                                          &                                   &                          \\ \hline
	\end{tabular}
\end{table}

\begin{itemize}
	\item \textbf{Proposer de formaliser la définition du concept (identifiables dans un corpus), mots clés ? Embeddings ? —>} 
	\item nous nous appuyons sur une approximation d'un tel concept, car la tâche d'automatisation et d'implémentation dans l'optique computationnelle enlève forcément quelques traits de concepts abordés dans ce chapitre
\end{itemize}





%\section{Études numériques des circulations culturelles}
\label{circulations}


%Les humanités numériques au service de l'analyse des circulations culturelles
%
%Comment définir une circulation du point de vue de l'analyse du texte ? de la linguistique computationnelle (TAL) ?

%\minitoc



\section{Extraction de la terminologie : un levier pour l'analyse de la diffusion scientifique ?}
\hl{Méthodo globale + SotA globale À COMPLÉTER}

%\section{Comment les mots deviennent-ils des concepts ?}