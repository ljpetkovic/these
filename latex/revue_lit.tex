\chapter{L’évolution des approches numériques pour détecter la circulation des savoirs}
\minitoc
\label{sect:sota_circulations}
Cette partie a pour objectif de dresser un panorama des projets de valorisation des archives textuelles ou audiovisuelles à l'aide d'outils numériques permettant d'identifier les différents formes de circulations culturelle ou intellectuelle. Nous y présentons des études menés au sein d'institutions patrimoniales et académiques, avant d'aborder les évènements et revues spécialisés sur la question des circulations.
\section{Principaux acteurs}
\label{sect:sota_acteurs} Incontestablement, l'époque actuelle est profondément marquée par le \og{}déluge des données\fg{}, phénomène représentatif de la quatrième paradigme de la science, selon Jim Gray \citep[p.~30]{hey2009jim}. Ce nouveau paradigme scientifique s'appuie sur la capacité des ordinateurs de recueillir, stocker et analyser rapidement d'importants volumes de données. Par conséquent, les projets numériques sont aujourd'hui largement \og{}pilotés par les données (angl. \textit{data-driven})\fg{}\footnote{Ce terme fut introduit par \citet{Johns1991ShouldYB}, dans l'expression \textit{data-driven learning}.}, et ceux qui sont centrés sur les explorations des circulations culturelles au prisme du numérique se concrétisent à grande échelle. Les méthodes computationnelles des circulations ne se limitent pas à la gestion de corpus volumineux selon une perspective spatio-temporelle ; elles permettent également de dépasser les cadres nationaux, de remettre en question les hiérarchies entre objets d’étude, et ainsi de s'affranchir des \og{}réflexes nationalisateurs\fg{} et des \og{}jugements de valeur\fg{} souvent présents dans l'analyse des circulations culturelles \citep[p.~12]{joyeux2022circulations}.
%Les humanités numériques au service de l'analyse des circulations culturelles se manifestent sous forme de divers projets de recherche au niveau académique. 
Cette thématique est au c\oe{}ur des préoccupations des structures de recherche (laboratoires, chaires, équipes-projet), des évènements scientifiques ou même des numéros de revues spécialisés.

En France, plusieurs équipes de recherche académique constituent un environnement dynamique consacré à l'étude des phénomènes de circulation. Le projet de recherche sous-tendant ce mémoire s'approche tangentiellement des travaux réalisés au sein de l'ancien labex \textsc{OBVIL}\footnote{\url{https://obtic.huma-num.fr/obvil-web/}} de Sorbonne Université, en particulier la thèse de \citet{riguet2018litterature} consacrée à la réception de la pensée scientifique du physiologiste français Claude Bernard dans la critique littéraire du \textsc{XIX}\ieme{} siècle. En s'appuyant sur des techniques d'alignement entre les textes de Bernard et des ouvrages de critique littéraire, ce travail met notamment en lumière des auteurs comme que Émile Zola, Émile Durkheim et Henri Bergson, qui reprennent les principes développés par ce scientifique \citep[p.~401]{riguet2018litterature}. L'équipe-projet \textsc{ObTIC}\footnote{\url{https://obtic.sorbonne-universite.fr/projets/}} poursuit l'héritage du labex \textsc{OBVIL} à travers le projet \textsc{ModERN}\footnote{\url{https://modern.huma-num.fr/about/}}, qui vise à identifier et analyser des réseaux conceptuels et intertextuels au sein d’une collection de textes des \textsc{XVIII}\ieme{} et \textsc{XIX}\ieme{} siècles. Au sein du même projet, nous notons les travaux de \citet{roe2023enlightenment} qui portent sur la détection de réemplois textuels à grande échelle et l'analyse de réseaux pour identifier les \og{}influenceurs\fg{} dans les ouvrages français du siècle des Lumières. Toujours au sein de Sorbonne Université, l'Alliance Sorbonne Université a lancé, le 1\textsuperscript{er} janvier 2025, le projet \og{}Circulations médiévales\fg{} (\textsc{MéCir})\footnote{\url{https://www.centrechastel.sorbonne-universite.fr/actualites-evenements/creation-de-linitiative-circulations-medievales-mecir}}, destiné à mobiliser de nouvelles technologies (humanités numériques, archéométrie\footnote{\og{}les recherches visant à appliquer à l'archéologie des techniques empruntées aux sciences expérimentales physico-chimiques ou biochimiques\fg{} \citep{bertrand2025archeometrie}.} et intelligence artificielle) pour analyser \og{}la transmission des savoirs, des textes, des biens et des cultures\fg{} tant au Moyen Âge qu'aux époques moderne et contemporaine.

D'autres universités françaises, comme l'École normale supérieure (\textsc{ENS-PSL}), ont développé les projets visant à valoriser les archives philosophiques au prisme des humanités numériques, notamment au sein de l'Observatoire des humanités numériques\footnote{\url{https://odhn.ens.psl.eu/}.}. Le projet \og{}Nietzsche et son temps\fg{}\footnote{\url{http://www.item.ens.fr/nietzsche/}} analyse la genèse des concepts et des thèmes principaux de la philosophie de Nietzsche à travers l'édition et à l’interprétation de son \oe{}uvre avec des méthodes génétiques. De manière similaire, le projet \og{}DERRIDA HEXADÉCIMAL\fg{}\footnote{\url{http://www.item.ens.fr/derrida-hexadecimal/}} [\textit{sic.}] s'est attaché à identifier les caractéristiques de l'invention scripturale propre à Derrida, en s'intéressant notamment à sa créativité linguistique (création de néologismes, mots-valise, jeux de signifiant,$\dots$), ainsi qu'à l'élaboration des concepts, en mobilisant pour cela des méthodes issues de la criminalistique numérique. Pour ce qui est du projet \og{}Foucault Fiches de lecture\fg{}\footnote{\url{https://eman-archives.org/Foucault-fiches/}}, il vise d'une part à rendre accessibles les sources utilisées par le philosophe, et d'autre part à contribuer à l'élaboration d'une herméneutique philosophique, fondée sur l'étude de ses pratiques documentaires et de ses modes de travail. La problématique de l'évaluation de l'importance d'une entité appartenant à un domaine ontologique occupe une place centrale dans le travail d'\citet{abdallah2025rankingdom} à l'université de Tours\footnote{\url{https://lifat.univ-tours.fr/lifat-english-version/teams/bdtln}}, ayant conduit au développement de l'outil de représentation des connaissances Rankingdom\footnote{\url{https://rankingdom.org/about}.}. L'un des volets de ce projet concerne les différentes déclinaisons de la notion d'importance d'une entité, associées à des métriques spécifiques telles que la portée (popularité), l'influence, \og{}à propos\fg{}, l'indice \textit{a} et l'impact, détaillées dans la partie \ref{resultats}. 

À l'échelle européenne et internationale, l'équipe meta\textsc{LAB}\footnote{\url{https://mlml.io/}} conduit le projet \textit{Peirce Interprets Peirce}\footnote{\url{mlml.io/p/peirce-interprets-peirce/}}, une étude computationnelle des manuscrits de Charles S. Peirce, figure majeure de la sémiotique moderne, dans le but d'analyser des motifs stylistiques et sémantiques et de retracer l'évolution de sa pensée scientifique. À ces initiatives académiques s'ajoutent les travaux menés par la chaire des humanités numériques de l'université de Genève\footnote{\url{https://www.unige.ch/lettres/humanites-numeriques/recherche/projets}}, qui s'articulent autour d'au moins trois projets : \textit{Artl@s}, \textit{Katabase -- Manuscripts Sales CatalogueS (MSS)} et \textit{Visual Contagions}. Le projet \textit{Artl@s}\footnote{\url{https://artlas.huma-num.fr/fr/}} vise à cartographier la circulation mondiale des \oe{}uvres d'art à partir des documents textuels, notamment les catalogues d'expositions. \textit{Katabase -- MSS}\footnote{\url{https://katabase.huma-num.fr/}}, quant à lui, explore la circulation des manuscrits en tant que nouvelle perspective pour analyser la réception des auteurs et, par conséquent, les évolutions du canon littéraire. À la différence de ces deux projets fondés sur l'exploitation des sources textuelles, \textit{Visual Contagions}\footnote{\url{https://www.unige.ch/visualcontagions/}} se penche sur la circulation mondiale des images à l'ère de l'imprimé, de 1890 jusqu'aux débuts d'Internet. Il convient également de mentionner une étude de faisabilité empirique axée sur l'annotation et la détection des réemplois de textes allusifs. Cette recherche s'intéresse à l'intertexte biblique dans les \oe{}uvres de Bernard de Clairvaux (1090-1153), écrivain médiéval majeur, reconnu pour l'omniprésence de ses référence à la Bible \citep{manjavacas}.


Le tableau \ref{tab:equipes} présente un récapitulatif des projets cités, classés par ordre alphabétique du$\cdot$e la responsable d’équipe. 
% \ref{tab:events} et \ref{tab:revues}.

\begin{table}[htbp]
	\centering
	\renewcommand{\arraystretch}{1.5}
	\resizebox{\textwidth}{!}{%
		\begin{tabular}[t]{|p{7cm}|p{7cm}|}
			\hline
			\rowcolor{blue!10}\multicolumn{1}{|c|}{\textbf{Équipe}} & \multicolumn{1}{c|}{\textbf{Projets numériques des circulations}} \\ \hline
			\begin{minipage}[t]{\linewidth}
				Labex \textsc{OBVIL} (transformé en \textsc{ObTIC})\\
				Sorbonne Université\\
				Responsable : Didier Alexandre
			\end{minipage}
			&
			\begin{minipage}[t]{\linewidth}
				\begin{itemize}[leftmargin=*]
					\item \textit{La Littérature laboratoire (1850-1914) : quand la critique littéraire défie la science} \citep{riguet2018litterature}
					
				\end{itemize}
			\end{minipage} \\ \hline
			\begin{minipage}[t]{\linewidth}
				Initiative Circulations médiévales (\textsc{MéCir}) | Alliance Sorbonne Université\\
				Responsable : Béatrice Caseau
			\end{minipage}
			&
			\begin{minipage}[t]{\linewidth}
				\begin{itemize}[leftmargin=*]
					\item Initiative Circulations médiévales (\textsc{MéCir}) 
				\end{itemize}
			\end{minipage} \\ \hline
			\begin{minipage}[t]{\linewidth}
				Chaire des humanités numériques\\
				Université de Genève\\
				Responsable : Béatrice Joyeux-Prunel
			\end{minipage}
			&
			\begin{minipage}[t]{\linewidth}
				\begin{itemize}[leftmargin=*]
					\item \textit{Artl@s} (2009-)
					\item \textit{Katabase -- Manuscripts Sales CatalogueS (MSS)} (2020-) 
					\item \textit{Visual Contagions} (2021-2024)
					
				\end{itemize}
			\end{minipage} \\ \hline
						\begin{minipage}[t]{\linewidth}
				Équipe-projet \textsc{ObTIC}\\
				Sorbonne Université\\
				Responsable : Glenn Roe
			\end{minipage}
			&
			\begin{minipage}[t]{\linewidth}
				\begin{itemize}[leftmargin=*]
					\item \textit{Mod\textsc{ERN}} (2022-)
					
				\end{itemize}
			\end{minipage} \\ \hline
			\begin{minipage}[t]{\linewidth}
				Observatoire des humanités numériques\\
				École normale supérieure | \textsc{PSL}\\
				Responsable : Léa Saint-Raymond
			\end{minipage} 
			& 
			\begin{minipage}[t]{\linewidth}
				\begin{itemize}[leftmargin=*]
					\item \og{}Nietzsche et son temps\fg{} (2009-)
					\item \og{}Foucault Fiches de Lecture \fg{} (2013-)
					\item \og{}DERRIDA HEXADÉCIMAL\fg{} (2018-2024)
				\end{itemize}
			\end{minipage} \\ 
					\begin{minipage}[t]{\linewidth}
				Bases de données et Traitement du langage naturel (\textsc{BdTln})\\
				Université de Tours\\
				Responsable : Arnaud Soulet
			\end{minipage}
			&
			\begin{minipage}[t]{\linewidth}
				\begin{itemize}[leftmargin=*]
					\item \textit{Rankingdom} 
					
				\end{itemize}
			\end{minipage} \\ \hline\hline
								\multicolumn{2}{|c|}{\textbf{Collaborations interuniversitaires}}\\ \hline
						\begin{minipage}[t]{\linewidth}
				meta\textsc{LAB}\\
				Université de Harvard | Université libre de Berlin | École de design de Bâle\\
				Responsable : Jeffrey Schnapp
			\end{minipage} 
			& 
			\begin{minipage}[t]{\linewidth}
				\begin{itemize}[leftmargin=*]
					\item \textit{Peirce Interprets Peirce} (2022-)
				\end{itemize}
			\end{minipage} \\ \hline
			\begin{minipage}[t]{\linewidth}
			Université d'Anvers, Computational Linguistics, Psycholinguistics and Sociolinguistics (\textsc{CLiPS}) |\\Université Notre-Dame
			\end{minipage} 
			& 
			\begin{minipage}[t]{\linewidth}
				\begin{itemize}[leftmargin=*]
					\item \textit{On the Feasibility of Automated Detection of Allusive Text Reuse} \citep{manjavacas}
				\end{itemize}
			\end{minipage} \\ \hline
		\end{tabular}%
	}
	\caption{Structures de recherche axées sur la thématique des circulations des savoirs.}
	\label{tab:equipes}
\end{table}



Le tableau \ref{tab:events} propose un aperçu des évènements consacrés à la thématique des circulations, classés par l'année. 

%
\begin{table}[htbp]
	\centering
	\resizebox{\textwidth}{!}{%
		\renewcommand{\arraystretch}{1.5}
		\begin{tabular}[t]{|p{3.5cm}|p{1.2cm}|p{5cm}|p{6cm}|}
			\hline 
			\rowcolor{blue!10}\multicolumn{1}{|c|}{\textbf{Type d'évènement}} & \multicolumn{1}{c|}{\textbf{Année}} & \multicolumn{1}{c|}{\textbf{Nom d'évènement}} & \multicolumn{1}{c|}{\textbf{Thématiques}}\\ \hline
			Colloque & 2025 &
			\begin{minipage}[t]{\linewidth}\og{}Le texte de l'autre. Dialogue interdisciplinaire autour de l'intertextualité et du discours rapporté\fg{}\footnote{\url{https://www.fabula.org/actualites/122618/le-texte-de-l-autre-dialogue-interdisciplinaire-autour-de-l-intertextualite-et-du-discours-rapporte.html}}
			\end{minipage}
			& Dialogues historiques des textes et des idées :
			\begin{itemize}
				\item études littéraires et historiques qui fondent leur existence sur la citation, l’évocation et l’analyse de textes antérieurs, sources premières de leurs réflexions
				\item études de l’histoire intellectuelle, l’histoire de l’éducation ou l’histoire religieuse, pour qui l’établissement de réseaux de citation(s) peut révéler la circulation d’idées, leur rejet ou leur acceptation
			\end{itemize} \\ \hline
			Colloque & 2023 &
			\begin{minipage}[t]{\linewidth}Humanistica 2023 \footnote{\url{https://humanistica2023.sciencesconf.org/}}
			\end{minipage}
			& \begin{itemize}
				\item migrations, mondialisation, transferts culturels, histoire transnationale
				\item histoire des objets
				\item circulations de motifs littéraires, visuels, sonores
				\item réemplois / emprunts / citations
			\end{itemize} \\ \hline
			Colloque & 2023 &
			\begin{minipage}[t]{\linewidth}\og{}De la transformation des sciences humaines par les humanités numériques\fg{} (ACFAS 2023)\footnote{\url{https://www.crihn.org/nouvelles/2022/12/11/colloque-de-la-transformation-des-sciences-humaines-par-les-humanites-numeriques-acfas-2023/}}
			\end{minipage} & 
			\begin{itemize}
				\item formes de production, circulation et validation des connaissances à l’époque du numérique dans le domaine des sciences humaines
			\end{itemize} \\ \hline
			Journée d'étude & 2021 &
			\begin{minipage}[t]{\linewidth}
				\og{}Circulation des écrits littéraires de la première modernité et humanités numériques\fg{}\footnote{\url{https://www.fabula.org/actualites/86846/circulation-des-ecrits-litteraires-de-la-premiere-modernite-et-humanites-numeriques.html}}
			\end{minipage}
			&
			\begin{minipage}[t]{\linewidth}
				\begin{itemize}[leftmargin=*]
					\item circulation des textes (intertextuelle, auctoriale, générique) 
					\item circulation des réseaux (entre auteurs, traducteurs, compilateurs, imprimeurs-libraires) qui animent la production et la diffusion des écrits littéraires de la première modernité ?
				\end{itemize}
			\end{minipage} \\ \hline
		\end{tabular}
	}
	\caption{Évènements scientifiques axés sur la thématique des circulations des savoirs.}
	\label{tab:events}
\end{table}

\begin{table}[htbp]
	\renewcommand{\arraystretch}{1.5}
	\resizebox{\textwidth}{!}{%
		\begin{tabular}[t]{|p{9cm}|p{5cm}|}
			\hline 
			\rowcolor{blue!10}\multicolumn{1}{|c|}{\textbf{Revue}} & \multicolumn{1}{c|}{\textbf{Thématiques}} \\ \hline
			\begin{minipage}[t]{\linewidth}
				\textit{Circulation des discours dans les récits complotistes, 130}, 2022\\
				Dir. : Valérie Bonnet, Arnaud Mercier et Gilles Siouffi\footnote{\textit{Cf.} les projets de la chaire : \url{https://journals.openedition.org/mots/30297}.}
			\end{minipage}
			&
			\begin{minipage}[t]{\linewidth}
				Circulations textuelles internationales du discours :
				\begin{itemize}[leftmargin=*]
					\item complotiste des \og Illuminati \fg{}  \citep{chaudet2022illuminati}
					\item \og conspirationniste \fg{} sur Twitter \citep{giry2022etudier} \item  antiféministe en ligne \citep{morin2022discours}
				\end{itemize}
			\end{minipage} \\ \hline
			&  \\ \hline
		\end{tabular}
	}
	\caption{Revues scientifiques axés sur la thématique des circulations des savoirs.}
	\label{tab:revues}
\end{table}


%\begin{enumerate}
%	\item certaines chaires universitaires, notamment celle des Humanités numériques à l'université de Genève \citep{joyeux2022circulations}\footnote{\textit{Cf.} les projets de la chaire : \url{https://www.unige.ch/lettres/humanites-numeriques/recherche/projets-de-la-chaire}.} ;
%	\item de divers évènements scientifiques, comme la journée d'étude \og{}Circulation des écrits littéraires de la première modernité et humanités numériques\fg{}\footnote{\url{https://www.fabula.org/actualites/86846/circulation-des-ecrits-litteraires-de-la-premiere-modernite-et-humanites-numeriques.html}}, les colloques Humanistica 2023\footnote{\url{https://humanistica2023.sciencesconf.org/}}, \textsc{ACFAS} 2023\footnote{\url{https://www.crihn.org/nouvelles/2022/12/11/colloque-de-la-transformation-des-sciences-humaines-par-les-humanites-numeriques-acfas-2023/}} etc. ;
%	\item des numéros de certaines revues, par exemple \og{}Circulation des discours dans les récits complotistes\fg{}, dont les articles portent sur les thématiques aussi diverses que les circulations textuelles internationales du discours complotiste des \og Illuminati \fg{}  \citep{chaudet2022illuminati}, \og conspirationniste \fg{} sur Twitter \citep{giry2022etudier} ou antiféministe en ligne \citep{morin2022discours}. 
%\end{enumerate}









%Ce mémoire est basé sur la contribution de \citet{petkovic2023circulation} s'inscrivant dans l'optique de l'exploration des circulations médicales. Nous souhaitons mesurer informatiquement l'impact scientifique des travaux de Charcot sur ses collaborateurs et successeurs, membres de son réseau scientifique. Cette mesure se fonde sur l'analyse des concepts-clés en matière de son discours scientifique, et plus particulièrement sur l'opérationnalisation du terme \og{}influence\fg{}, définie ici comme une intertextualité\footnote{Nous nous appuyons sur la définition de l'intertextualité dans la littérature, où ce terme désigne \og{}la perception, par le lecteur, de rapports entre une \oe{}uvre et d'autres, qui l'ont précédée ou suivie\fg{} \citep[p.~4]{riffaterre1980trace}.} uni-directionnelle, allant des écrits de Charcot (ci-après corpus \og{}Charcot\fg{}) vers ceux de ses collaborateurs et successeurs (ci-après corpus \og{}Autres\fg{}). Il s'agit donc \textit{in fine} d'aborder computationnellement la question des circulations, non pas des artefacts matériels comme les manuscrits \citep{gabay2021katabase} et les images \citep{joyeux2019visual}, mais des phénomènes textuels complexes \citep{manjavacas} ayant une dimension théorique forte. 



\label{sect:sota}
\begin{table}[h]
	\centering
	\begin{tabularx}{\textwidth}{|>{\centering\arraybackslash}p{3.5cm}|>{\centering\arraybackslash}X|>{\centering\arraybackslash}X|}
		\hline
		Type & Titre &  \\
		\hline
	\end{tabularx}
\end{table}

\hl{SotA...}
\url{https://savoirs.app/}

\url{https://docs.google.com/document/d/1eoW3mDiHYB9vrPtG-5pdPuaUAAUJpWDa/edit\#heading=h.gjdgxs}


Étant donné l'importance des travaux de Charcot et ses contributions dans le domaine de la neurologie et au-delà, nous souhaitons explorer la notion de la circulation des savoirs au prisme du numérique à travers son impact. Avant d'aborder la question d'opérationnalisation de son impact, nous tenons d'abord à décortiquer les mécanismes à l'origine des circulations des savoirs à grande échelle, ainsi que de définir la notion d'un \og{}concept\fg{} pouvant véhiculer les informations importantes concernant les circulations en question. 





%La section \ref{concept} élabore les différentes approches pour définir plus globalement la notion des concepts historiques qui nous orienteront vers une définition des concepts médicaux en particulier. \smalltodo{pont}



