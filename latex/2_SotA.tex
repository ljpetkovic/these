\section{Comment les mots deviennent-ils des concepts ?}
\label{concept}

Afin de pouvoir analyser les concepts médicaux liés à Charcot, il est important de déterminer de quelle manière un mot ou un groupe de mots devient un concept général ou scientifique. Les termes \textit{idée}, \textit{concept}, \textit{terme}, \textit{mot} et \textit{mot-clé} figurent parmi des notions fondamentales dans les disciplines aussi théoriques (linguistique générale, épistémologie ou philosophie) que numériques ou celles ayant un aspect appliqué, comme p. ex. \textsc{TAL} et \textsc{HN}. 
Malgré leur présence répandue dans les domaines cités, ainsi que leur utilisation devenue quasi banale dans le langage courant, ces notions demeurent sans définition fixe et universellement acceptée en raison de la disparité des contextes dans lesquels elles sont utilisées. En plus, elles sont interdépendantes et la frontière entre eux est floue. 

Concernant la notion du concept, quelques remarques philosophiques de \citeauthor{Lecourt1999} (\citeyear{Lecourt1999}, p.~261-263) méritent d'être soulignées ici. Premièrement, l'invention de l'entité du concept remonte à l'ère d'Aristote, qui l'a caractérisé comme une abstraction, un mode de connaissance médiat et général, et comme mode de classification entre le genre et l'espèce (\textit{intension} et \textit{extension}, respectivement). L'intension du concept de chat est sa définition : \og{}animal à quatre pattes de la famille des félins\fg{}, tandis que son extension est un chat concret : le chat tigré, mon chat etc. Deuxièmement, un concept décrit un sujet, il est définissable et représente un résultat de l'abstraction du donné\footnote{Le concept de \og{}donné\fg{} est utilisé en philosophie pour désigner \og{}ce qui est immédiatement présent à l'esprit avant que celui-ci n'y applique ses procédés d'élaboration\fg{}, \url{http://stella.atilf.fr/}.} empirique qui forme une de ses extensions. Cette notion n'est pas à confondre avec celle de l'\textit{idée}, qui représente elle-même l'objet de connaissance et la condition même du concept, distinction faite de manière systématique chez Kant. Finalement, au-delà des définitions du concept présentées ci-dessus du point de vue phénoménologique à travers l'intension et l'extension, la notion du concept peut également être comprise comme un élément d'un jugement qui peut être une loi scientifique. En d'autres mots, la conception d'un concept inclut non seulement les descriptions d'un sujet en utilisant les prédicats à une place (a.), mais s'étend aussi aux relations \textit{n}-aires (b.) ou même à celles entre des concepts plus abstraits qui impliquent des propriétés allant au-delà des simples prédicats (c.). Cette théorie plus \og{}inférentielle\fg{} est à l'origine des concepts scientifiques, dont l'illustration nous retrouvons dans les exemples suivants :
\begin{itemize}
	\item[\quad (a.)] \og{}le chat est roux\fg{} : \textit{le chat} est un sujet (concept) et \textit{être roux} est un prédicat ;
	\item[\quad (b.)] \og{}le chat voit un chien\fg{} : le sujet \textit{le chat} forme une relation binaire avec un objet \textit{un chien} à l'aide du prédicat \textit{voir} ;
	\item[\quad (c.)] \og{}Dans un \textit{triangle rectangle}, le \textit{carré} de la \textit{longueur} de l'\textit{hypoténuse} est \textit{égal} à la \textit{somme} des \textit{carrés} des \textit{longueurs} des deux \textit{côtés} de l'\textit{angle droit}\fg{} : les concepts mathématiques sont typographiés en italique.
\end{itemize}
\medskip

Nous juxtaposons ce point de vue aux réflexions sociohistoriques de \citeauthor{stengers1987d} (\citeyear{stengers1987d}) qui rendent compte des particularités des concepts scientifiques. D'après elle, l'attribut \textit{scientifique} est associé à leur objectivité et leur puissance explicative, or il n'implique pour autant pas une neutralité d'avis qui est considérée néfaste pour les recherches scientifiques et en même temps fictive. L'autrice renforce cette idée en prétendant que le concept scientifique est forcément controversé, puisqu'il est sujet aux discussions, aux polémiques et aux consensus, ce qui impose une prise de position. Le concept scientifique a des rôles particuliers dans les opérations régissant un champ scientifique, notamment sa singularité, son pouvoir d'extension et d'organisation effective des phénomènes, en s'opposant ainsi à la simple présentation des idées de la part de son$\cdot$sa émetteur$\cdot$trice, tout en comprenant un aspect polémique \citep[pp.~10-11]{stengers1987d}. 

À ces traits s'ajoute celui que la même autrice appelle \og{}la propagation épidémique\fg{} (p.~16), où les domaines \og{}infectés\fg{} par un concept scientifique peuvent être autonomes et devenir une source de nouvelle propagation. Cela est illustré sur l'exemple du concept \og programme \fg{} en biologie (matériel génétique et sa fonction) qui a migré vers le domaine de l'informatique (opération d'un ordinateur). Les concepts sont donc capables de voyager d'une science à l'autre, ce qui a inspiré la métaphore des \og{}concepts nomades\fg{}, marqués par leur circulation spatio-temporelle et linguistique. Outre la nature itinérante des concepts scientifiques qui contribue à l'interdisciplinarité et à la production des savoirs nouveaux, \citeauthor{stengers1987d} (\citeyear{stengers1987d}, pp.~21-23) se réfère aux opérations de la \og capture \fg{} de la scientificité par ces concepts et du \og durcissement \fg{} conséquent des sciences. À savoir, certains concepts atteignent le degré de maturité après s'être avéré être adéquats et pertinents dans les démarches scientifiques dont ils \og{}capturent\fg{} la scientificité, permettant ainsi que le statut des sciences se solidifie ou \og durcisse \fg{}. La capture implique la définition, mais aussi la redéfinition d'une notion par les spécialistes d'une science. Les points de vue de \citeauthor{stengers1987d} (\citeyear{stengers1987d}) relèvent de la théorie constructiviste du savoir scientifique, selon laquelle la science est une \og{}construction\fg{} collective issue du contexte socio-historique (p. ex. interaction entre les scientifiques, les institutions etc.), et non pas d'une accumulation neutre et objective de faits.

Cette approche est complémentaire à l'histoire des concepts (allem. \textit{Begriffsgeschichte}), dans laquelle les significations des concepts en général sont considérées d'être les dérivés d'un contexte sociopolitique. Plus précisément, cette transformation d'un ou plusieurs mots en un concept survient lorsque cette construction linguistique comprend toute la gamme des significations dérivées d'un tel contexte \citep[p.~19]{koselleck2011introduction}. À titre d'exemple, le concept d'un \textit{état} ne peut être interprété qu'à travers ses différents constituants, dont \textit{souveraineté territoriale, législation, fiscalité}, parmi maints d'autres. L'histoire des concepts concerne principalement les manifestations de conflits sociopolitiques particuliers qui doivent être compris dans leur contexte approprié, où p. ex. les mots comme \textit{liberté} ou \textit{démocratie} portent la connotation polémique dont le sens ne peut être précisé qu'à travers leurs antithèses (\textit{esclavage} et \textit{dictature}, respectivement). Les concepts sont donc les concentrations par défaut ambiguës d'une multitude de contenus sémantiques, uniquement interprétables et indéfinissables, par contraste avec des significations des mots qui peuvent être définies de manière exacte \citep[p. 20]{koselleck2011introduction}. De plus, les concepts comme \textit{histoire} ou \textit{progrès} sont caractérisés comme \og{}collectifs singuliers\fg{} qui marquent un passage du domain concret d'un individu (plusieurs \textit{histoires} et \textit{progrès} individuels) au domain abstrait et général du collectif social (une \textit{histoire} ou un \textit{progrès} général ou collectif). Ce phénomène linguistique, ainsi que la création des concepts comme \textit{industrie, usine, classe moyenne} etc., reflète un changement de paradigme dans l'organisation sociale survenu lors des révolutions politiques et industrielles \citep[p. 1]{hobsbawm2010age}. La période charnière concernée par ce phénomène est nommée \textit{Sattelzeit} (allem. \og{}époque de selle\fg{}), entre 1750 et 1850, durant laquelle les concepts historiques deviennent abstraits, singularisés, respatialisés et retemporalisés \citep[pp.~34-35]{koselleck2011introduction}. Cela traduit le lien fort entre l'histoire du langage et l'histoire des idées.

Ces considérations sont appliquables à d'autres \og{}concepts nomades\fg{} en sciences humaines et sociales (ci-après \textsc{SHS}), comme \textit{travail}, \textit{intelligencija}, \textit{Ancien Régime}, \textit{avant-garde}, \textit{Occident} etc. qui font partie du \textit{Dictionnaire des concepts nomades en sciences humaines} \citep{christin2011dictionnaire}. Plusieurs questionnements ont été soulevés par \citeauthor{ghermani2011} (\citeyear{ghermani2011}, p.~117) eu égard de leur émergence, notamment pour déterminer à quel moment un concept devient une entrée dans un dictionnaire des \textsc{SHS} : \og{}\textit{Pourquoi un concept fait-il son entrée dans un dictionnaire ? Au terme de quel processus ? À l'inverse, comment cette percée lexicale est-elle parfois impossible ou refusée ?}\fg{}. Contrairement aux processus de la propagation et de la capture qui permettaient à un concept d'obtenir le statut de scientificité, l'autrice souligne les pratiques scientifiques conduisant aux rétractations et aux masquages de sens des concepts en \textsc{SHS}, p. ex. dans le cas du terme \og{}confession [religieuse]\fg{}, dont le sens varie en fonction de l'historiographie dans laquelle il figure \citep[p.~117]{ghermani2011}. Enfin, \citeauthor{bal2002travelling} (\citeyear{bal2002travelling}, p.~34) va plus loin en excluant la \og diffusion \fg{} et en mettant en avant la \og propagation \fg{} comme le critère discriminatoire de la nature itinérante des concepts. 

Pour résumer la complexité de la définition des concepts du point de vue de leur histoire, nous citons ici \citeauthor{bal2002travelling} (\citeyear{bal2002travelling}, p.~51), selon laquelle les concepts sont :
\begin{itemize}
	\item datés, et donc marqués par une évolution ;
	\item les mots : archaïsmes et néologismes relevant des mécanismes étymologiques qui leur donnent une dimension philosophique ;
	\item syntaxiques au sein d'une langue ;
	\item en évolution constante ;
	\item créés, et non pas donnés \textit{a priori}.
\end{itemize}
\medskip
Concernant plus précisément le concept scientifique, l'épistémologie en esquisse les traits suivants, comme souligné par \citeauthor{rumelhard1986} (\citeyear{rumelhard1986}) et cité dans \citeauthor{astolfi2008chapitre} (\citeyear{astolfi2008chapitre}, p.~25) :
\begin{itemize}
	\item le concept scientifique possède une dénomination et une définition, avec le sens le plus univoque possible, \textit{a contrario} du concept linguistique, en principe équivoque et polysémique ;
	\item fonction opératoire : le concept scientifique est un outil intellectuel, un instrument théorique permettant d'interpréter des phénomènes ;
	\item fonction d'opérateur, caractérisé par son degré de formalisation et par les interconnexions avec les techniques scientifiques ;
	\item une extension, une compréhension, un domaine et des limites de validités en lien étroit avec sa définition fixée ;
	\item le concept scientifique peut être compris comme un n\oe{}ud dans un réseau de relations organisé, au sein duquel il dialogue avec d'autres concepts et théories scientifiques.
\end{itemize}

\hl{Ajouter une définition du concept du point de vue de TAL}

\section{Extraction de la terminologie : un levier pour l'analyse de la diffusion scientifique ?}
\hl{une section à part ? quel titre ? comment l'articuler ? Méthodo globale + SotA globale}

