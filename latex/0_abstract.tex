

%\begin{abstract}
%\hskip7mm
%%
%\begin{spacing}{1.3}
Ce projet de thèse, à la jonction des Lettres (histoire des sciences) et de l'informatique, propose une étude interdisciplinaire centrée sur la valorisation du fonds patrimonial de Jean-Martin Charcot, fondateur de la neurologie moderne au XIX\ieme{} siècle en France, au prisme des humanités numériques (\textsc{HN}) et du traitement automatique des langues (\textsc{TAL}). Plus concrètement, cette recherche se concentre sur l'exploration de la circulation des savoirs, à travers les reprises des théories scientifiques de Charcot sous forme de concepts médicaux dans ses propres écrits et dans ceux d'autres scientifiques. Ce travail vise à établir un protocole permettant d'appréhender la circulation de concepts de manière automatisée. La première contribution de ce travail est la constitution d'un corpus numérique à partir des archives en question déjà numérisées. Comme deuxième contribution, nous proposons la mise en place d'une chaîne de traitement semi-automatique consistant en l'océrisation, la correction de sortie OCR, la structuration des données au format XML-TEI, la fouille sémantique et l'alignement des textes pour étudier le transfert interdisciplinaire du discours médical de Charcot dans les écrits réalisés en collaboration et dans ceux de ses continuateurs et disciples. Ces traitements nous permettront de produire une transcription interrogeable dans notre cadre de recherche. Au-delà des finalités de ce projet de thèse, ce modèle généralisable sera aussi applicable à d'autres projets de numérisation et de valorisation des fonds patrimoniaux.

\textbf{Mots-clés} : Jean-Martin Charcot ; humanités numériques ; traitement automatique des langues ; extraction des phrases-clés.

%Au-delà du cas de Charcot, ce travail vise à établir un protocole permettant d'appréhender la circulation de concepts de manière automatisée.

\centredchapter{Abstract}
%
%\end{spacing}
%\end{abstract}
