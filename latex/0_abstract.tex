%\renewcommand{\abstractnamefont}{\normalfont\Large\bfseries}
%\renewcommand{\abstracttextfont}{\normalfont\Huge}

%\begin{abstract}
%\hskip7mm
%
%\begin{spacing}{1.3}

Ce projet de thèse propose une étude interdisciplinaire centrée sur la valorisation du fonds patrimonial de Jean-Martin Charcot, fondateur de la neurologie moderne au XIX\textsuperscript{e} siècle en France, au prisme des humanités numériques (\textsc{HN}) et du traitement automatique des langues (\textsc{TAL}). Plus concrètement, cette recherche se concentre sur l'exploration de la circulation des savoirs, à travers les reprises du discours scientifique de Charcot sous forme de concepts médicaux dans les écrits d'autres scientifiques. Le présent mémoire vise également à approfondir les recherches issues du travail de \citet{petkovic2023circulation} s'inscrivant dans l'optique de l'exploration quantitative de ce type de circulation. 


Au-delà du cas de Charcot, ce travail vise à établir un protocole permettant d'appréhender la circulation de concepts de manière automatisée.

\textbf{Mots-clés} : Jean-Martin Charcot ; humanités numériques ; traitement automatique des langues ; champs lexicaux.
%\end{spacing}
%\end{abstract}
