\minitoc% Creating an actual minitoc

%\begin{itemize}
%\item 10-60 pages (10\% du total de la thèse)
%\end{itemize}

%Feuille de route :
%\begin{enumerate}
%\item accroche
%\item présentation globale du sujet
%\item SotA, cadre théorique de la thèse
%\item problématique
%\item hypothèses
%\item méthodo du recueil de données
%\item motivations, objectif de l'étude, portée
%\item annonce du plan
%\end{enumerate}

\begin{itemize}
\item présentation du sujet et de la problématique
\end{itemize}

Ce projet de thèse propose une étude interdisciplinaire avec le focus sur la valorisation numérique du fonds patrimonial de Jean-Martin Charcot, fondateur de la neurologie moderne au XIX\ieme{} siècle en France.
Le présent mémoire est basé sur la contribution de \citet{petkovic2023circulation} s'inscrivant dans l'optique de l'exploration quantitative des circulations des concepts médicaux. Ainsi, cette thématique, au carrefour de l'histoire des sciences et de la linguistique computationnelle, fait partie des travaux de l'axe \#3 qui sont menés par l'équipe-projet \textsc{ObTIC}\footnote{\url{https://obtic.sorbonne-universite.fr/presentation/}.}. 
Notre objectif de thèse est double :
\begin{itemize} 
\item formaliser une approche pour tracer l'évolution des concepts médicaux en général ;
\item en prenant comme cas d'étude les archives de Charcot, pister numériquement la circulation de son discours médical.
\end{itemize}

Nous nous intéressons tout particulièrement à l'analyse de la genèse et de la migration du discours médical de la pathologie anatomique, de la neurologie et de la psychologie dans les écrits de Charcot réalisés en collaboration et dans ceux de ses disciples et continuateurs. Si l'importance des contributions scientifiques de Charcot est un sujet largement étudié du point de vue théorique (\citealp{bogousslavsky2011following,broussolle2012,camargo2024} ), cet aspect reste inexploré dans une perspective quantitative.

À ce titre, nous visons à mesurer informatiquement l'impact des travaux de Charcot sur son réseau scientifique\footnote{Nous engloberons sous ce terme l'ensemble de ses collègues et successeurs.}. Cette mesure se fonde sur l'analyse des concepts-clés en matière de son discours scientifique, et plus particulièrement sur l'opérationnalisation du terme \og{}influence\fg{}, définie ici comme une intertextualité uni-directionnelle, allant des écrits de Charcot (ci-après corpus \og{}Charcot\fg{}) vers ceux de ses collaborateurs et successeurs (ci-après corpus \og{}Autres\fg{}). Les objectifs de cette thèse consistaient donc à formaliser la définition de la notion des concepts scientifiques identifiables dans notre corpus d'étude, et à proposer une démarche de l'exploration de ces concepts du point de vue numérique. Il s'agit donc \textit{in fine} d'aborder computationnellement la question des circulations, non pas des artefacts matériels comme les manuscrits \citep{gabay2021katabase} et les images \citep{joyeux2019visual}, mais des phénomènes textuels complexes \citep{manjavacas} ayant une dimension théorique forte. Cela nous permettra d'y analyser également le discours médical de Charcot à travers l'extraction des expressions à mots multiples \citep[p. 96]{nerima2006}\footnote{angl. \textit{multi-word expressions}, se déclinant sous la forme suivante, entre autres : \textsc{substantif + adjectif + adjectif}. Exemple : la pathologie \textit{sclérose latérale amyotrophique}.}, qui constituent potentiellement des champs lexicaux et des savoirs en circulation.

Ce mémoire est structuré en cinq parties principales : après l'introduction, nous esquissons l'évolution des théories scientifiques dans une perspective épistémologique, en prenant comme cas d'étude les contributions majeures de Charcot (chapitre \ref{circulations}).
Ensuite, nous proposons une revue de la littérature portant sur les modalités des circulations des objets patrimoniaux du point de vue numérique (chapitre \ref{sota}). Le chapitre \ref{corpus} donne un aperçu de la constitution du corpus de recherche. Le chapitre \ref{resultats} présente les premières tentatives de l'analyse computationnelle de l'impact de Charcot sur son réseau scientifique, ainsi que les limites de ces approches, en proposant une nouvelle méthode pour la quantification de la pertinence des expressions polylexicales. Enfin, le chapitre \ref{conclusion} propose une conclusion et des pistes pour des recherches futures.

\section{SotA, cadre théorique de la thèse}
...
\section{Problématique}
...
\section{Hypothèses}
...
\section{Méthodo du recueil de données}
...
\section{Motivations, objectif de l'étude, portée + annonce du plan}
...






