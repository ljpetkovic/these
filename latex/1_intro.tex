\minitoc% Creating an actual minitoc

%\begin{itemize}
%\item 10-60 pages (10\% du total de la thèse)
%\end{itemize}

%Feuille de route :
%\begin{enumerate}
%\item accroche
%\item présentation globale du sujet
%\item SotA, cadre théorique de la thèse
%\item problématique
%\item hypothèses
%\item méthodo du recueil de données
%\item motivations, objectif de l'étude, portée
%\item annonce du plan
%\end{enumerate}


%\section{Enjeux et motivations}
%\begin{itemize}
%\item présentation du sujet et de la problématique
%\end{itemize}



L’intérêt pour ce parcours doctoral s'ancre dans les expériences de l'autrice en valorisation numérique de ressources textuelles variées, menées lors du master en \og Informatique pour les sciences humaines \fg{}\footnote{\url{http://arhiva.rect.bg.ac.rs/en/education/interdisciplinary/computing.php}} à l'université de Belgrade et du certificat de spécialisation en linguistique\footnote{\url{https://www.unige.ch/lettres/linguistique/program/postgrade}} à Genève. Ces travaux ont porté sur des corpus tels que les paroles de chansons rock d’ex-Yougoslavie \citep{petkovic2019creation}, les manuscrits de M\textsuperscript{me} de Sévigné (\citealp{gabay2020quantifying,gabay2021katabase})\footnote{Projet \textit{Katabase} : \url{https://katabase.huma-num.fr/}}, ou encore les catalogues d’expositions\footnote{Projet \textit{Virtual Contagions} : \url{https://www.unige.ch/visualcontagions/}}.
À ces projets s'ajoute le stage effectué au sein de l'équipe-projet Observatoire des Textes, des Idées et des Corpus (\textsc{ObTIC})\footnote{\url{https://obtic.sorbonne-universite.fr/presentation/}.} de Sorbonne Université entre le 29 mars 2021 et le 31 juillet 2021, sous l'encadrement de Prof. D\textsuperscript{r} Glenn Roe et D\textsuperscript{r} Motasem Alrahabi, ingénieur de recherche. Dans le cadre du projet de la Très Grande Bibliothèque (\textsc{TGB})\footnote{\url{http://obvil.lip6.fr/tgb}}, l'enjeu de ce stage a été d'exploiter le corpus constitué d'un grand volume de documents \textsc{XML-TEI} en français, OCRisés\footnote{Forme francisée dérivée de l'abréviation anglaise \textsc{OCR} pour \textit{optical character recognition}, soit \og{}reconnaissance optique de caractères\fg{}.} (transcrits automatiquement) et non corrigés, issus des collections Gallica de la Bibliothèque nationale de France (\textsc{BnF}). 
À l'issue du stage, le projet de recherche doctoral a été mis en place lors de la campagne d'attribution de contrats doctoraux 2021 par l'institut Observatoire des patrimoines de l'Alliance Sorbonne Université (\textsc{OPUS})\footnote{\url{https://institut-opus.sorbonne-universite.fr/node/478}}. 

Ce projet vise à répondre aux enjeux portés par l'\textsc{OPUS}, ainsi qu'à ceux de la Bibliothèque de neurosciences Jean-Martin Charcot -- rattachée à la Bibliothèque de Sorbonne Université (\textsc{BSU})\footnote{\url{https://www.sorbonne-universite.fr/bu/decouvrir-nos-bibliotheques/la-bibliotheque-charcot}} et détentrice du fonds patrimonial de Jean-Martin Charcot -- dans une perspective de valorisation de ce fonds. Plus globalement, cette thèse entend s'inscrire dans les dynamiques de la communauté des \textsc{HN}, dont les travaux portent sur les objets patrimoniaux sous toutes leurs formes : (im)matériels, culturels ou naturels. Au carrefour de l'histoire des sciences et de la linguistique computationnelle, ce projet fait également partie des travaux de l'axe \#3 qui sont menés par l'équipe-projet \textsc{ObTIC} et qui sont tournés vers le dialogue constant entre les recherches qualitatives et quantitatives. 

La motivation à poursuivre ce travail de recherche découle d’une part de l'abondance des études consacrées à la valorisation des archives patrimoniales et aux circulations culturelles (voir section~\ref{sect:sota_valorisation}). Elle s'appuie également sur la confusion que suscite le terme \textit{circulation des savoirs}, souvent méconnu des chercheur$\cdot$se$\cdot$x$\cdot$s ne travaillant pas dans le domaine du \textsc{TAL} ou des \textsc{HN}.
Cette barrière disciplinaire est apparue de manière particulièrement tangible lors des échanges de l’autrice avec les spécialistes de Charcot en médecine, histoire des sciences et critique littéraire. C'est pourquoi cette thèse entend constituer un pas vers la démystification des méthodes quantitatives, mises au service des recherches en histoire des sciences et plus largement en sciences humaines et sociales (ci-après \og{}\textsc{SHS}\fg{}).

\section{Un projet pour pister numériquement la circulation des savoirs}
Ce projet de thèse propose une étude interdisciplinaire centrée sur la valorisation du fonds patrimonial de Jean-Martin Charcot, fondateur de la neurologie moderne au XIX\textsuperscript{e} siècle en France, au prisme des humanités numériques (\textsc{HN}) et du traitement automatique des langues (\textsc{TAL}). Nous nous intéressons à l'analyse de la genèse et de la migration des savoirs médicaux de Charcot en pathologie anatomique, neurologie et psychologie. Plus concrètement, cette recherche se concentre sur l'exploration de la circulation des savoirs à travers les reprises des théories scientifiques de Charcot sous forme de concepts médicaux dans les écrits co-rédigés par Charcot ainsi que dans ceux de ses disciples, collaborateurs et successeurs constituant son \og{}réseau scientifique\fg{}. Le présent mémoire vise également à approfondir les recherches issues du travail de \citet{petkovic2023circulation} s'inscrivant dans l'optique de l'exploration quantitative de ce type de circulation. 
 Si l'importance des contributions scientifiques de Charcot est un sujet largement étudié du point de vue de l'histoire des neurosciences (\citealp{bogousslavsky2011following,broussolle2012,camargo2024}, parmi de nombreux autres travaux), cet aspect reste inexploré dans une perspective quantitative. Cela n'est pas très étonnant, étant donné que l'étude des textes dans les archives en ligne reste un domaine en cours de développement dans un contexte de circulation des connaissances \citep[p.~4]{milia2023}.
%Dans le cadre de cette thèse, nous nous intéressons à la circulation à travers l'analyse de la genèse et de la migration du discours médical -- en pathologie anatomique, neurologie et psychologie -- dans les écrits co-rédigés par Charcot ainsi que dans ceux de ses disciples, collaborateurs et successeurs constituant son \og{}réseau scientifique\fg{}. Si l'importance des contributions scientifiques de Charcot est un sujet largement étudié du point de vue de l'histoire des neurosciences (\citealp{bogousslavsky2011following,broussolle2012,camargo2024}, parmi de nombreux autres travaux), cet aspect reste inexploré dans une perspective quantitative. Cela n'est pas très étonnant, étant donné que l'étude des textes dans les archives Internet et des données en ligne dans un contexte de circulation des connaissances en général reste un domaine en cours de développement \citep{milia2023}. À ce titre, nous nous tâchons à mesurer informatiquement l'impact des travaux de Charcot sur son réseau scientifique. Cette mesure se fonde sur l'analyse des concepts-clés en matière de son discours scientifique, et plus particulièrement sur l'opérationnalisation du terme \og{}influence\fg{}, définie ici comme une intertextualité uni-directionnelle, allant des écrits de Charcot (ci-après corpus \og{}Charcot\fg{}) vers ceux de son réseau scientifique (ci-après corpus \og{}Autres\fg{}). Il s'agit donc \textit{in fine} d'aborder computationnellement la question des circulations, non pas des artefacts matériels comme les manuscrits \citep{gabay2021katabase} et les images \citep{joyeux2019visual}, mais des phénomènes textuels complexes \citep{manjavacas} ayant une dimension théorique forte. 


Dans le cadre de l'analyse numérique de l'impact scientifique de Charcot, nous étudions \textit{in fine} la circulation de ses théories et des concepts médicaux dont il était inventeur (p. ex. \textit{sclérose latérale amyotrophique} -- \textit{SLA}) et transmetteur (p. ex. \textit{hystérie})\footnote{En effet, Charcot n'a pas inventé ce terme, mais en réinterprété le sens. Pour une discussion détaillée sur l'évolution de ce terme, voir la partie \ref{hysterie}.}. Cette démarche nous oblige de :
\begin{enumerate}
	\item formaliser en premier lieu la définition du terme \textit{concept scientifique}, identifiable dans un corpus numérique, tout en prenant en compte les difficultés inhérentes à la définition d'un concept \textit{per se}, ainsi qu'à celle de ses termes apparentés : \textit{idée}, \textit{terme}, \textit{mot} ou \textit{mot-clé} (parties \ref{concept} et \ref{termes}) ;
	\item comprendre, conceptualiser et opérationnaliser \og{}comment des concepts, des théories ou des méthodes circulent, s'échangent, s'empruntent, se transfèrent et se transforment dans le passage d'une discipline à une autre\fg{}, questionnement partagé avec \citeauthor{landais2014frederic} (\citeyear{landais2014frederic}, p.~331) (partie \ref{sect:modalites_circulations}).
\end{enumerate}

%Le pré-requis pour analyser ce type de circulation est de formaliser les concepts scientifiques identifiables dans notre corpus d'étude.
Cela est intrinsèquement lié au double objectif de cette thèse, car nous souhaitons :
\begin{itemize} 
	\item formaliser une approche numérique pour tracer l'évolution des concepts médicaux en général ;
	\item en prenant comme cas d'étude les archives de Charcot, pister numériquement la circulation des savoirs médicaux dans la communauté scientifique de son époque.
\end{itemize}
\medskip


%\section{Enjeux et motivations}
%\begin{itemize}
%\item présentation du sujet et de la problématique
%\end{itemize}






\section{La complexité du terme \og{}circulation des savoirs\fg{}}
\label{sect:modalites_circulations}
Les savoirs participent à un double mouvement d'héritage et de transmission. En effet, leur circulation sur le temps long reflète ces dynamiques de transmission, essentielles à la formation de courants de pensée, ainsi qu'à l'affirmation d’une identité construite autour d'un savoir partagé \citep[p.~251]{adell2011chapitre}. Dans une perspective contemporaine, de nombreux$\cdot$ses chercheur$\cdot$se$\cdot$s$\cdot$x partagent le point de vue selon lequel la notion de circulation des savoirs constitue un champ de recherche vaste, ainsi qu'un nouveau paradigme de la connaissance depuis le début du XXI\ieme{} siècle et l'avènement du Web \textsc{2.0} (\citealp{landais2014frederic,quet2014frederic}). Cette phase de l'évolution du Web se caractérisait notamment par la transformation majeure de l'Internet en vue du développement des réseaux sociaux, des blogs et des sites participatifs, tout en permettant aux utilisateur$\cdot$trice$\cdot$s$\cdot$x de créer, partager et interagir avec du contenu Web. Nous traversons actuellement l'ère du Web \textsc{3.0}, né dans les années 2010 et appelé également \og{}Web sémantique\fg{}, qui permet de lier et structurer l'information afin d'en extraire la connaissance (\citeauthor{andrade2013sociologie} \citeyear{andrade2013sociologie}, p.~107). Par ailleurs, \citeauthor{landais2014frederic} (\citeyear{landais2014frederic}, p.~331) remarque que ce type de circulation connaît une croissance importante grâce aux outils de la numérisation de la production scientifique et de l'édition numérique des ouvrages.
%les savoirs sont amenés à circuler, à voyager, à se propager, mais aussi à communiquer plus rapidement ce qui ouvre de nombreuses pistes de recherche, aussi bien théoriques qu'appliquées, orientées vers l'exploration de la nature de ces circulations. 

Le terme en question reste toutefois assez complexe en raison de visions différentes sur la façon de le définir. Afin d'éclairer cette problématique, \citeauthor{quet2014frederic} (\citeyear{quet2014frederic}, pp.~221--222) souligne trois aspects suivants :
\begin{enumerate}
	\item \textbf{Éléments de la circulation}. Qu'est-ce qui circule ? 
	\begin{itemize}
		\item individus (savants, techniciens, traducteurs, etc.) ;
		\item objets matériels (instruments scientifiques, ouvrages etc.) :
		\item constructions symboliques (théories, concepts etc.).
	\end{itemize}  
	\item \textbf{Conceptions de la circulation et méthodes de son analyse} ;
	\begin{itemize}
		\item définition de la circulation comme \og{}traduction\fg{}, \og{}diffusion\fg{}, \og{}accès\fg{} ou \og{}succès\fg{} ;
		\item critères méthodologiques possibles pour étudier la circulation p. ex. d'une théorie : 
		\begin{itemize}
			\item circulations géographiques des principaux concepteurs qu'on lui reconnaît ;
			\item circulations et lectures des textes produits par leurs concepteurs ;
			\item usages et applications analogiques qui en sont faits dans d'autres domaines.
		\end{itemize} 
		\item enjeux d'articulation de ces différents niveaux d'observation du point de vue méthodologique et de celui de la production du texte de recherche, dans le cas des croisements de ces niveaux.
	\end{itemize}
	\item \textbf{Conceptions analytiques et normatives des savoirs}
	\begin{itemize}
		\item affaiblissement des catégories des \og{}savoirs profanes\fg{} et \og{}savoirs scientifiques\fg{}, ainsi que de l'opposition entre eux ;
		\item revalorisation des savoirs implicites et de la dimension pratique des connaissances ;
		\item glorification de la circulation comme porteuse de valeurs \textit{a priori} positives : confrontation à l'autre, hybridation, production de nouveauté, etc.
	\end{itemize}
\end{enumerate}


%\section{Modalités des circulations des savoirs}
%De nombreux$\cdot$ses chercheur$\cdot$se$\cdot$s$\cdot$x partagent le point de vue selon lequel la notion de circulation des savoirs constitue un champ de recherche vaste, ainsi qu'un nouveau paradigme de la connaissance depuis le début du XXI\ieme{} siècle et l'avènement du Web \textsc{2.0} (\citealp{landais2014frederic,quet2014frederic}). Cette phase de l'évolution du Web se caractérisait notamment par la transformation majeure de l'Internet en vue du développement des réseaux sociaux, des blogs et des sites participatifs, tout en permettant aux utilisateur$\cdot$trice$\cdot$s$\cdot$x de créer, partager et interagir avec du contenu Web. Nous traversons actuellement l'ère du Web \textsc{3.0}, né dans les années 2010 et appelé également \og{}Web sémantique\fg{}, qui permet de lier et structurer l'information afin d'en extraire la connaissance (\citeauthor{andrade2013sociologie} \citeyear{andrade2013sociologie}, p.~107). Néanmoins, en se référant à la circulation des savoirs, \citeauthor{landais2014frederic} (\citeyear{landais2014frederic}, p.~331) remarque que ce phénomène connaît une croissance importante grâce aux outils de la numérisation de la production scientifique et de l'édition numérique des ouvrages.
%%les savoirs sont amenés à circuler, à voyager, à se propager, mais aussi à communiquer plus rapidement ce qui ouvre de nombreuses pistes de recherche, aussi bien théoriques qu'appliquées, orientées vers l'exploration de la nature de ces circulations. 
%
%Le terme en question reste toutefois assez complexe en raison de visions différentes sur la façon de le définir. Afin d'éclairer cette problématique, \citeauthor{quet2014frederic} (\citeyear{quet2014frederic}, pp.~221--222) souligne trois aspects suivants :
%\begin{enumerate}
%	\item \textbf{Éléments de la circulation}. Qu'est-ce qui circule ? 
%	\begin{itemize}
%		\item individus (savants, techniciens, traducteurs, etc.) ;
%		\item objets matériels (instruments scientifiques, ouvrages etc.) :
%		\item constructions symboliques (théories, concepts etc.).
%	\end{itemize}  
%	\item \textbf{Conceptions de la circulation et méthodes de son analyse} ;
%	\begin{itemize}
%		\item définition de la circulation comme \og{}traduction\fg{}, \og{}diffusion\fg{}, \og{}accès\fg{} ou \og{}succès\fg{} ;
%		\item critères méthodologiques possibles pour étudier la circulation p. ex. d'une théorie : 
%		\begin{itemize}
%			\item circulations géographiques des principaux concepteurs qu'on lui reconnaît ;
%			\item circulations et lectures des textes produits par leurs concepteurs ;
%			\item usages et applications analogiques qui en sont faits dans d'autres domaines.
%		\end{itemize} 
%		\item enjeux d'articulation de ces différents niveaux d'observation du point de vue méthodologique et de celui de la production du texte de recherche, dans le cas des croisements de ces niveaux.
%	\end{itemize}
%	\item \textbf{Conceptions analytiques et normatives des savoirs}
%	\begin{itemize}
%		\item affaiblissement des catégories des \og{}savoirs profanes\fg{} et \og{}savoirs scientifiques\fg{}, ainsi que de l'opposition entre eux ;
%		\item revalorisation des savoirs implicites et de la dimension pratique des connaissances ;
%		\item glorification de la circulation comme porteuse de valeurs \textit{a priori} positives : confrontation à l'autre, hybridation, production de nouveauté, etc.
%	\end{itemize}
%\end{enumerate}
%
%Dans le cadre de l'analyse numérique de l'impact scientifique de Charcot, nous étudions \textit{in fine} la circulation de ses théories et des concepts médicaux dont il était inventeur (p. ex. \textit{SLA}) et transmetteur (p. ex. \textit{hystérie})\footnote{Comme déjà expliqué dans la partie \ref{hysterie}, Charcot n'a pas inventé ce terme, mais en réinterprété le sens.}. Cette démarche nous oblige de :
%\begin{enumerate}
%	\item formaliser en premier lieu la définition du terme \textit{concept scientifique}, identifiable dans un corpus numérique, tout en prenant en compte les difficultés inhérentes à la définition d'un concept \textit{per se}, ainsi qu'à celle de ses termes apparentés : \textit{idée}, \textit{terme}, \textit{mot} ou \textit{mot-clé} (parties \ref{concept} et \ref{termes}) ;
%	\item comprendre, conceptualiser et opérationnaliser \og{}comment des concepts, des théories ou des méthodes circulent, s'échangent, s'empruntent, se transfèrent et se transforment dans le passage d'une discipline à une autre\fg{}, questionnement partagé avec \citeauthor{landais2014frederic} (\citeyear{landais2014frederic}, p.~331) (partie \ref{circulations}).
%\end{enumerate}

Dans le cadre de notre étude, le terme \og{}circulation des savoirs\fg{} est entendu comme la diffusion des constructions symboliques (théories et concepts scientifiques) produites par leur concepteur (Charcot) et observées dans les ouvrages de son réseau. Ainsi, le terme \og{}diffusion\fg{} englobe l'influence \textit{a priori} positive (\og{}succès\fg{}, \og{}production de nouveauté\fg{}), mais aussi la réception parfois mitigée des observations de Charcot.

%\section{Modalités des circulations des savoirs}
%De nombreux$\cdot$ses chercheur$\cdot$se$\cdot$s$\cdot$x partagent le point de vue selon lequel la notion de circulation des savoirs constitue un champ de recherche vaste, ainsi qu'un nouveau paradigme de la connaissance depuis le début du XXI\ieme{} siècle et l'avènement du Web \textsc{2.0} (\citealp{landais2014frederic,quet2014frederic}). Cette phase de l'évolution du Web se caractérisait notamment par la transformation majeure de l'Internet en vue du développement des réseaux sociaux, des blogs et des sites participatifs, tout en permettant aux utilisateur$\cdot$trice$\cdot$s$\cdot$x de créer, partager et interagir avec du contenu Web. Nous traversons actuellement l'ère du Web \textsc{3.0}, né dans les années 2010 et appelé également \og{}Web sémantique\fg{}, qui permet de lier et structurer l'information afin d'en extraire la connaissance (\citeauthor{andrade2013sociologie} \citeyear{andrade2013sociologie}, p.~107). Néanmoins, en se référant à la circulation des savoirs, \citeauthor{landais2014frederic} (\citeyear{landais2014frederic}, p.~331) remarque que ce phénomène connaît une croissance importante grâce aux outils de la numérisation de la production scientifique et de l'édition numérique des ouvrages.
%%les savoirs sont amenés à circuler, à voyager, à se propager, mais aussi à communiquer plus rapidement ce qui ouvre de nombreuses pistes de recherche, aussi bien théoriques qu'appliquées, orientées vers l'exploration de la nature de ces circulations. 
%
%Le terme en question reste toutefois assez complexe en raison de visions différentes sur la façon de le définir. Afin d'éclairer cette problématique, \citeauthor{quet2014frederic} (\citeyear{quet2014frederic}, pp.~221--222) souligne trois aspects suivants :
%\begin{enumerate}
%	\item \textbf{Éléments de la circulation}. Qu'est-ce qui circule ? 
%	\begin{itemize}
%		\item individus (savants, techniciens, traducteurs, etc.) ;
%		\item objets matériels (instruments scientifiques, ouvrages etc.) :
%		\item constructions symboliques (théories, concepts etc.).
%	\end{itemize}  
%	\item \textbf{Conceptions de la circulation et méthodes de son analyse} ;
%	\begin{itemize}
%		\item définition de la circulation comme \og{}traduction\fg{}, \og{}diffusion\fg{}, \og{}accès\fg{} ou \og{}succès\fg{} ;
%		\item critères méthodologiques possibles pour étudier la circulation p. ex. d'une théorie : 
%		\begin{itemize}
%			\item circulations géographiques des principaux concepteurs qu'on lui reconnaît ;
%			\item circulations et lectures des textes produits par leurs concepteurs ;
%			\item usages et applications analogiques qui en sont faits dans d'autres domaines.
%		\end{itemize} 
%		\item enjeux d'articulation de ces différents niveaux d'observation du point de vue méthodologique et de celui de la production du texte de recherche, dans le cas des croisements de ces niveaux.
%	\end{itemize}
%	\item \textbf{Conceptions analytiques et normatives des savoirs}
%	\begin{itemize}
%		\item affaiblissement des catégories des \og{}savoirs profanes\fg{} et \og{}savoirs scientifiques\fg{}, ainsi que de l'opposition entre eux ;
%		\item revalorisation des savoirs implicites et de la dimension pratique des connaissances ;
%		\item glorification de la circulation comme porteuse de valeurs \textit{a priori} positives : confrontation à l'autre, hybridation, production de nouveauté, etc.
%	\end{itemize}
%\end{enumerate}
%
%Dans le cadre de l'analyse numérique de l'impact scientifique de Charcot, nous étudions \textit{in fine} la circulation de ses théories et des concepts médicaux dont il était inventeur (p. ex. \textit{SLA}) et transmetteur (p. ex. \textit{hystérie})\footnote{Comme déjà expliqué dans la partie \ref{hysterie}, Charcot n'a pas inventé ce terme, mais en réinterprété le sens.}. Cette démarche nous oblige de :
%\begin{enumerate}
%	\item formaliser en premier lieu la définition du terme \textit{concept scientifique}, identifiable dans un corpus numérique, tout en prenant en compte les difficultés inhérentes à la définition d'un concept \textit{per se}, ainsi qu'à celle de ses termes apparentés : \textit{idée}, \textit{terme}, \textit{mot} ou \textit{mot-clé} (parties \ref{concept} et \ref{termes}) ;
%	\item comprendre, conceptualiser et opérationnaliser \og{}comment des concepts, des théories ou des méthodes circulent, s'échangent, s'empruntent, se transfèrent et se transforment dans le passage d'une discipline à une autre\fg{}, questionnement partagé avec \citeauthor{landais2014frederic} (\citeyear{landais2014frederic}, p.~331) (partie \ref{circulations}).
%\end{enumerate}



