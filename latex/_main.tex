\documentclass[a4paper]{report}
% \counterwithout{figure}{chapter} % pour compter la Figure 1 au lieu de Figure 1.1
%====================== PACKAGES ======================
\usepackage[round,authoryear]{natbib}
\usepackage[greek.ancient,french]{babel}
\usepackage{fontspec}
\babelfont[greek]{rm}
          [Scale=MatchLowercase]{Gentium Plus}
\usepackage[LGR,T1]{fontenc}
\usepackage{lmodern}
\usepackage{gentium}
\usepackage{xcolor}
\frenchsetup{StandardItemLabels=true}
%pour gérer les positionnement d'images
\usepackage{float}
\usepackage{amsmath}
\usepackage{graphicx}
\usepackage[colorinlistoftodos]{todonotes}
\usepackage{url}
%pour les informations sur un document compilé en PDF et les liens externes / internes
\usepackage{hyperref}
%pour la mise en page des tableaux
\usepackage{array}
\usepackage{tabularx}
%pour utiliser \floatbarrier
%\usepackage{placeins}
%\usepackage{floatrow}
%espacement entre les lignes
\usepackage{setspace}
%modifier la mise en page de l'abstract
\usepackage{abstract}
%police et mise en page (marges) du document
%\usepackage{polyglossia}
%\setmainlanguage{french}
%\setotherlanguage[variant=polytonic]{greek}
%\newfontfamily{\greekfont}[Ligatures=TeX]{Linux Libertine O}
%\usepackage{fontspec}
%\setmainfont{DejaVu Serif}
\usepackage[top=2cm, bottom=2cm, left=3cm, right=3cm]{geometry}
%Pour les galerie d'images
\usepackage{subfig}
\usepackage{fancyhdr}
\pagestyle{fancy}
\fancyhf{}
\fancyhead[L]{\rightmark}
\fancyhead[R]{\thepage}
\renewcommand{\headrulewidth}{0.4pt}% Default \headrulewidth is 0.4pt
\renewcommand{\footrulewidth}{0.4pt}% Default \footrulewidth is 0pt

% renommer « Fig. » en « Figure »
\addto\captionsfrench{\renewcommand{\figurename}{\textsc{Figure}}}

\usepackage{lipsum} 
\usepackage{blindtext, color}
\usepackage[explicit]{titlesec}
\definecolor{gray75}{gray}{0.35}
%\newcommand{\hsp}{\hspace{20pt}}
%\titleformat{\chapter}[hang]{\Huge\bfseries}{Chapitre \thechapter\hsp\textcolor{gray75}{|}\hsp}{0pt}{\Huge\bfseries}

% How can I have a vertical bar in chapter title which scales with the broken title?
\newcommand{\hsp}{\hspace{20pt}}
\titleformat{\chapter}[hang]{\huge\bfseries}{\color{gray75}Chapitre \thechapter}{20pt}{\begin{tabular}[t]{@{\color{gray75}\vrule width 2pt\hsp}p{0.75\textwidth}}\raggedright#1\end{tabular}}

\usepackage{gensymb} % pour le symbole du degré (\degree)

\usepackage[
    left = \flqq{},% 
    right = \frqq{},% 
    leftsub = \flq{},% 
    rightsub = \frq{} %
]{dirtytalk}

%====================== INFORMATION ET REGLES ======================

%rajouter les numérotation pour les \paragraphe et \subparagraphe
\setcounter{secnumdepth}{4}
\setcounter{tocdepth}{4}

% ajouter de l'espace entre le numéro de la footnote et le texte de la footnote
\usepackage[hang]{footmisc}
\setlength{\footnotemargin}{2mm}

\usepackage{etoolbox}
\gappto{\UrlBreaks}{\UrlOrds}
\hypersetup{							% Information sur le document
pdfauthor = {Premier Auteur,
			Deuxième Auteur,
			Troisième Auteur,
    		Quatrième Auteur},			% Auteurs
pdftitle = {Nom du Projet -
			Sujet du Projet},			% Titre du document
pdfsubject = {Mémoire de Projet},		% Sujet
pdfkeywords = {Tag1, Tag2, Tag3, ...},	% Mots-clefs
pdfstartview={FitH},
    colorlinks=true,
    linkcolor=black,
    filecolor=magenta,      
    urlcolor=blue,
    citecolor=black,
    linkbordercolor=red,
    citebordercolor=red,
    urlbordercolor=red,
    pdftitle={Overleaf Example},
    pdfpagemode=FullScreen}					% ajuste la page à la largueur de l'écran
%pdfcreator = {MikTeX},% Logiciel qui a crée le document
%pdfproducer = {}} % Société avec produit le logiciel
\usepackage{fancyhdr}% http://ctan.org/pkg/fancyhdr
% \pagestyle{fancy}% Change page style to fancy
% \fancyhf{}% Clear header/footer
% \fancyhead[C]{}
% \fancyfoot[C]{}% \fancyfoot[R]{\thepage}

\newenvironment{myepigraph}
  {\par\hfill\itshape
   \begin{tabular}{@{}r@{\hspace{2em}}}} % 2em from the right margin
  {\end{tabular}\par\medskip}
  

%======================== DEBUT DU DOCUMENT ========================

\begin{document}

%régler l'espacement entre les lignes
\newcommand{\HRule}{\rule{\linewidth}{0.5mm}}

%page de garde
%\input{title.tex} % ici mettre la page de garde de SU

%page blanche
\newpage
~
%ne pas numéroter cette page
\thispagestyle{empty}
\newpage
\setcounter{page}{0}
\input{remerciements}
\thispagestyle{empty}
\setcounter{page}{0}
%ne pas numéroter le sommaire
\newpage
~
\thispagestyle{empty}
\setcounter{page}{0}
%ne pas numéroter le sommaire

\newpage

%\renewcommand{\abstractnamefont}{\normalfont\Large\bfseries}
%\renewcommand{\abstracttextfont}{\normalfont\Huge}

%\begin{abstract}
%\hskip7mm
%
%\begin{spacing}{1.3}


Ce projet de thèse propose une étude interdisciplinaire centrée sur la valorisation du fonds patrimonial de Jean-Martin Charcot, fondateur de la neurologie moderne au XIX\textsuperscript{e} siècle en France, au prisme des humanités numériques (\textsc{HN}) et du traitement automatique des langues (\textsc{TAL}). Plus concrètement, cette recherche se concentre sur l'exploration de la circulation des savoirs, à travers les reprises du discours scientifique de Charcot sous forme de concepts médicaux dans les écrits d'autres scientifiques. Le présent mémoire vise également à approfondir les recherches issues du travail de \citet{petkovic2023circulation} s'inscrivant dans l'optique de l'exploration quantitative de ce type de circulation. 


Au-delà du cas de Charcot, ce travail vise à établir un protocole permettant d'appréhender la circulation de concepts de manière automatisée.


\textbf{Mots-clés} : Jean-Martin Charcot ; humanités numériques ; traitement automatique des langues ; extraction des phrases-clés.
%\end{spacing}
%\end{abstract}

\thispagestyle{empty}
\setcounter{page}{0}
%ne pas numéroter le sommaire
\newpage
~
\thispagestyle{empty}
\setcounter{page}{0}
%ne pas numéroter le sommaire

\newpage

\tableofcontents
\thispagestyle{empty}
\setcounter{page}{0}
%ne pas numéroter le sommaire

\newpage

%espacement entre les lignes d'un tableau
\renewcommand{\arraystretch}{1.5}

%====================== INCLUSION DES PARTIES ======================

~
\thispagestyle{empty}
%recommencer la numérotation des pages à "1"
\setcounter{page}{0}
\newpage



\chapter{Introduction}

\chapter{La rupture épistémologique en médecine : la notion d'\textit{hystérie}}
\minitoc% Creating an actual minitoc

%\begin{itemize}
%\item 10-60 pages (10\% du total de la thèse)
%\end{itemize}

%Feuille de route :
%\begin{enumerate}
%\item accroche
%\item présentation globale du sujet
%\item SotA, cadre théorique de la thèse
%\item problématique
%\item hypothèses
%\item méthodo du recueil de données
%\item motivations, objectif de l'étude, portée
%\item annonce du plan
%\end{enumerate}
\section{Motivation}
%\begin{itemize}
%\item présentation du sujet et de la problématique
%\end{itemize}

Ce projet de thèse propose une étude interdisciplinaire avec le focus sur la valorisation numérique du fonds patrimonial de Jean-Martin Charcot, fondateur de la neurologie moderne au XIX\ieme{} siècle en France. 
L'intérêt de poursuivre ce parcours doctoral puise ses racines dans les participations de l'autrice de ce mémoire aux travaux portant sur l'exploration numérique des ressources textuelles aussi diverses que les paroles de chansons rock d'ex-Yougoslavie \citep{petkovic2019creation} et des manuscrits de M\textsuperscript{me} de Sévigné (\citealp{gabay2020quantifying,gabay2021katabase}). 
À ces projets s'ajoute également le stage effectué au sein de l'équipe-projet Observatoire des Textes, des Idées et des Corpus (\textsc{ObTIC})\footnote{\url{https://obtic.sorbonne-universite.fr/presentation/}.} de Sorbonne Université entre le 29 mars 2021 et le 31 juillet 2021, sous l'encadrement de Prof. D\textsuperscript{r} Glenn Roe et D\textsuperscript{r} Motasem Alrahabi, ingénieur de recherche. Dans le cadre du projet de la Très Grande Bibliothèque (\textsc{TGB}), l'enjeu de ce stage a été d'exploiter le corpus constitué d'un grand volume de documents \textsc{XML} en français, océrisés\footnote{Forme francisée dérivée de l'abréviation anglaise \textsc{OCR} pour \textit{optical character recognition}, soit \og{}reconnaissance optique de caractères\fg{}.} (transcrits automatiquement) et non corrigés, issus des collections Gallica de la Bibliothèque nationale de France (\textsc{BnF}). 
À l'issue du stage, le projet de recherche doctoral a été mis en place grâce à la campagne d'attribution de contrats doctoraux 2021 par l'institut Observatoire des patrimoines de l'Alliance Sorbonne Université (\textsc{OPUS})\footnote{\url{https://institut-opus.sorbonne-universite.fr/node/478}}. Il a pour objectif de répondre aux enjeux de l'institut, mais aussi à ceux de la communauté des humanités numériques, dont les projets traitent des objets patrimoniaux sous toutes leurs formes : (im)matériels, culturels ou naturels.  

Le présent mémoire vise à approfondir les recherches issues du travail de \citet{petkovic2023circulation} s'inscrivant dans l'optique de l'exploration quantitative des circulations des concepts médicaux. Ainsi, cette thématique, au carrefour de l'histoire des sciences et de la linguistique computationnelle, fait partie des travaux de l'axe \#3 qui sont menés par l'équipe-projet \textsc{ObTIC}. Nous nous intéressons tout particulièrement à l'analyse de la genèse et de la migration du discours médical de la pathologie anatomique, de la neurologie et de la psychologie dans les écrits de Charcot réalisés en collaboration et dans ceux de ses disciples et continuateurs. Si l'importance des contributions scientifiques de Charcot est un sujet largement étudié du point de vue théorique (\citealp{bogousslavsky2011following,broussolle2012,camargo2024} ), cet aspect reste inexploré dans une perspective quantitative.

À ce titre, nous nous tâchons à mesurer informatiquement l'impact des travaux de Charcot sur son réseau scientifique\footnote{Nous engloberons sous ce terme l'ensemble de ses collègues et successeurs.}. Cette mesure se fonde sur l'analyse des concepts-clés en matière de son discours scientifique, et plus particulièrement sur l'opérationnalisation du terme \og{}influence\fg{}, définie ici comme une intertextualité uni-directionnelle, allant des écrits de Charcot (ci-après corpus \og{}Charcot\fg{}) vers ceux de ses collaborateurs et successeurs (ci-après corpus \og{}Autres\fg{}). Les objectifs de cette thèse consistaient donc à formaliser la définition de la notion des concepts scientifiques identifiables dans notre corpus d'étude, et à proposer une démarche de l'exploration de ces concepts du point de vue numérique. Il s'agit donc \textit{in fine} d'aborder computationnellement la question des circulations, non pas des artefacts matériels comme les manuscrits \citep{gabay2021katabase} et les images \citep{joyeux2019visual}, mais des phénomènes textuels complexes \citep{manjavacas} ayant une dimension théorique forte. Cela nous permettra d'y analyser également le discours médical de Charcot à travers l'extraction des expressions à mots multiples \citep[p. 96]{nerima2006}\footnote{angl. \textit{multi-word expressions}, se déclinant sous la forme suivante, entre autres : \textsc{substantif + adjectif + adjectif}. Exemple : la pathologie \textit{sclérose latérale amyotrophique}.}, qui constituent potentiellement des champs lexicaux et des savoirs en circulation.






\section{Valorisation des archives patrimoniales et recherches quantitatives des circulations des savoirs : état actuel}
...SotA, cadre théorique de la thèse

\url{https://docs.google.com/document/d/1eoW3mDiHYB9vrPtG-5pdPuaUAAUJpWDa/edit\#heading=h.gjdgxs}
\section{Problématique + méthodo du recueil de données + hypothèses}
...
\section{Objectif de l'étude, portée + annonce du plan}
Notre objectif de thèse est double, car nous souhaitons :
\begin{itemize} 
	\item formaliser une approche pour tracer l'évolution des concepts médicaux en général ;
	\item en prenant comme cas d'étude les archives de Charcot, pister numériquement la circulation de son discours médical.
\end{itemize}

...

Ce mémoire est structuré en cinq parties principales : après l'introduction, nous esquissons l'évolution des théories scientifiques dans une perspective épistémologique, en prenant comme cas d'étude les contributions majeures de Charcot (chapitre \ref{circulations}).
Ensuite, nous proposons une revue de la littérature portant sur les modalités des circulations des objets patrimoniaux du point de vue numérique (chapitre \ref{sota}). Le chapitre \ref{corpus} donne un aperçu de la constitution du corpus de recherche. Le chapitre \ref{resultats} présente les premières tentatives de l'analyse computationnelle de l'impact de Charcot sur son réseau scientifique, ainsi que les limites de ces approches, en proposant une nouvelle méthode pour la quantification de la pertinence des expressions polylexicales. Enfin, le chapitre \ref{conclusion} propose une conclusion et des pistes pour des recherches futures.







\chapter{Circulations numériques}
\label{sota}
\minitoc
\section{Modalités des circulations des savoirs}
De nombreux$\cdot$ses chercheur$\cdot$se$\cdot$s$\cdot$x partagent le point de vue selon lequel la notion de circulation des savoirs constitue un champ de recherche vaste, ainsi qu'un nouveau paradigme de la connaissance depuis le début du XXI\ieme{} siècle et l'avènement du Web \textsc{2.0} (\citealp{landais2014frederic,quet2014frederic}). Cette phase de l'évolution du Web se caractérisait notamment par la transformation majeure de l'Internet en vue du développement des réseaux sociaux, des blogs et des sites participatifs, tout en permettant aux utilisateur$\cdot$trice$\cdot$s$\cdot$x de créer, partager et interagir avec du contenu Web. Nous traversons actuellement l'ère du Web \textsc{3.0}, né dans les années 2010 et appelé également \og{}Web sémantique\fg{}, qui permet de lier et structurer l'information afin d'en extraire la connaissance (\citeauthor{andrade2013sociologie} \citeyear{andrade2013sociologie}, p.~107). Néanmoins, en parlant de la circulation des savoirs, \citeauthor{landais2014frederic} (\citeyear{landais2014frederic}, p.~331) remarque que ce phénomène connaît une croissance importante grâce aux outils de la numérisation de la production scientifique et de l'édition numérique des ouvrages.
%les savoirs sont amenés à circuler, à voyager, à se propager, mais aussi à communiquer plus rapidement ce qui ouvre de nombreuses pistes de recherche, aussi bien théoriques qu'appliquées, orientées vers l'exploration de la nature de ces circulations. 

Le terme en question reste toutefois assez complexe en raison de visions différentes sur la façon de le définir. Afin d'éclairer cette problématique, \citeauthor{quet2014frederic} (\citeyear{quet2014frederic}, pp.~221--222) souligne trois aspects suivants :
\begin{enumerate}
    \item \textbf{Éléments de la circulation}. Qu'est-ce qui circule ? 
    \begin{itemize}
        \item individus (savants, techniciens, traducteurs, etc.) ;
        \item objets matériels (instruments scientifiques, ouvrages etc.) :
        \item constructions symboliques (théories, concepts etc.).
    \end{itemize}  
    \item \textbf{Conceptions de la circulation et méthodes de son analyse} ;
    \begin{itemize}
        \item définition de la circulation comme \og{}traduction\fg{}, \og{}diffusion\fg{}, \og{}accès\fg{} ou \og{}succès\fg{} ;
        \item critères méthodologiques possibles pour étudier la circulation p. ex. d'une théorie : 
        \begin{itemize}
            \item circulations géographiques des principaux concepteurs qu'on lui reconnaît ;
            \item circulations et lectures des textes produits par leurs concepteurs ;
            \item usages et applications analogiques qui en sont faits dans d'autres domaines.
        \end{itemize} 
        \item enjeux d'articulation de ces différents niveaux d'observation du point de vue méthodologique et de celui de la production du texte de recherche, dans le cas des croisements de ces niveaux.
    \end{itemize}
    \item \textbf{Conceptions analytiques et normatives des savoirs}
    \begin{itemize}
        \item affaiblissement des catégories des \og{}savoirs profanes\fg{} et \og{}savoirs scientifiques\fg{}, ainsi que de l'opposition entre eux ;
        \item revalorisation des savoirs implicites et de la dimension pratique des connaissances ;
        \item glorification de la circulation comme porteuse de valeurs \textit{a priori} positives : confrontation à l'autre, hybridation, production de nouveauté, etc.
    \end{itemize}
\end{enumerate}

Dans le cadre de l'analyse numérique de l'impact scientifique de Charcot, nous étudions \textit{in fine} la circulation de ses théories et des concepts médicaux dont il était inventeur (p. ex. \textit{SLA}) et transmetteur (p. ex. \textit{hystérie})\footnote{Comme déjà expliqué dans la partie \ref{hysterie}, Charcot n'a pas inventé ce terme, mais en réinterprété le sens.}. Cette démarche nous oblige de :
\begin{enumerate}
\item formaliser en premier lieu la définition du terme \textit{concept scientifique}, identifiable dans un corpus numérique, tout en prenant en compte les difficultés inhérentes à la définition d'un concept \textit{per se}, ainsi qu'à celle de ses termes apparentés : \textit{idée}, \textit{terme}, \textit{mot} ou \textit{mot-clé} (partie \ref{concept}) ;
\item comprendre, conceptualiser et opérationnaliser \og{}comment des concepts, des théories ou des méthodes circulent, s'échangent, s'empruntent, se transfèrent et se transforment dans le passage d'une discipline à une autre\fg{}, questionnement partagé avec \citeauthor{landais2014frederic} (\citeyear{landais2014frederic}, p.~331) (partie \ref{circulations}).
\end{enumerate}


%La section \ref{concept} élabore les différentes approches pour définir plus globalement la notion des concepts historiques qui nous orienteront vers une définition des concepts médicaux en particulier. \smalltodo{pont}

\section{Comment les mots deviennent-ils des concepts ?}
\label{concept}

Afin de pouvoir analyser les concepts médicaux liés à Charcot, il est important de déterminer de quelle manière un mot ou un groupe de mots devient un concept général ou scientifique. Les termes \textit{idée}, \textit{concept}, \textit{terme}, \textit{mot} et \textit{mot-clé} figurent parmi des notions fondamentales dans les disciplines aussi théoriques (linguistique générale, épistémologie ou philosophie) que numériques ou celles ayant un aspect appliqué, comme p. ex. traitement automatique des langues (TAL) et humanités numériques. 
Malgré leur présence répandue dans les domaines cités, ainsi que leur utilisation devenue quasi banale dans le langage courant, ces notions demeurent sans définition fixe et universellement acceptée en raison de la disparité des contextes dans lesquels elles sont utilisées. En plus, elles sont interdépendantes et la frontière entre eux est floue. 

Concernant la notion du concept, quelques remarques philosophiques de \citeauthor{Lecourt1999} (\citeyear{Lecourt1999}, p.~261-263) méritent d'être soulignées ici. Premièrement, l'invention de l'entité du concept remonte à l'ère d'Aristote, qui l'a caractérisé comme une abstraction, un mode de connaissance médiat et général, et comme mode de classification entre le genre et l'espèce (\textit{intension} et \textit{extension}, respectivement). L'intension du concept de chat est sa définition : \og{}animal à quatre pattes de la famille des félins\fg{}, tandis que son extension est un chat concret : le chat tigré, mon chat etc. Deuxièmement, un concept décrit un sujet, il est définissable et représente un résultat de l'abstraction du donné\footnote{Le concept de \og{}donné\fg{} est utilisé en philosophie pour désigner \og{}ce qui est immédiatement présent à l'esprit avant que celui-ci n'y applique ses procédés d'élaboration\fg{}, \url{http://stella.atilf.fr/Dendien/scripts/tlfiv5/advanced.exe?8;s=2289545040;.}} empirique qui forme une de ses extensions. Cette notion n'est pas à confondre avec celle de l'\textit{idée}, qui représente elle-même l'objet de connaissance et la condition même du concept, distinction faite de manière systématique chez Kant. Finalement, au-delà des définitions du concept présentées ci-dessus du point de vue phénoménologique à travers l'intension et l'extension, la notion du concept peut également être comprise comme un élément d'un jugement qui peut être une loi scientifique. En d'autres mots, la conception d'un concept inclut non seulement les descriptions d'un sujet en utilisant les prédicats à une place (a.), mais s'étend aussi aux relations \textit{n}-aires (b.) ou même à celles entre des concepts plus abstraits qui impliquent des propriétés allant au-delà des simples prédicats (c.). Cette théorie plus \og{}inférentielle\fg{} est à l'origine des concepts scientifiques, dont l'illustration nous retrouvons dans les exemples suivants :
\begin{itemize}
\item[\quad (a.)] \og{}le chat est roux\fg{} : \textit{le chat} est un sujet (concept) est et \textit{être roux} est un prédicat ;
\item[\quad (b.)] \og{}le chat voit un chien\fg{} : le sujet \textit{le chat} forme une relation binaire avec un objet \textit{un chien} à l'aide du prédicat \textit{voir} ;
\item[\quad (c.)] \og{}Dans un \textit{triangle rectangle}, le \textit{carré} de la \textit{longueur} de l'\textit{hypoténuse} est \textit{égal} à la \textit{somme} des \textit{carrés} des \textit{longueurs} des deux \textit{côtés} de l'\textit{angle droit}\fg{} : les concepts mathématiques sont typographiés en italique.
\end{itemize}
\medskip

Nous juxtaposons ce point de vue aux réflexions sociohistoriques de \citeauthor{stengers1987d} (\citeyear{stengers1987d}) qui rendent compte des particularités des concepts scientifiques. D'après elle, l'attribut \textit{scientifique} est associé à leur objectivité et leur puissance explicative, or il n'implique pour autant pas une neutralité d'avis qui est considérée néfaste pour les recherches scientifiques et en même temps fictive. L'autrice renforce cette idée en prétendant que le concept scientifique est forcément controversé, puisqu'il est sujet aux discussions, aux polémiques et aux consensus, ce qui impose une prise de position. Le concept scientifique a des rôles particuliers dans les opérations régissant un champ scientifique, notamment sa singularité, son pouvoir d'extension et d'organisation effective des phénomènes, en s'opposant ainsi à la simple présentation des idées de la part de son$\cdot$sa émetteur$\cdot$trice, tout en comprenant un aspect polémique \citep[pp.~10-11]{stengers1987d}. 

À ces traits s'ajoute celui que la même autrice appelle \og{}la propagation épidémique\fg{} (p.~16), où les domaines \og{}infectés\fg{} par un concept scientifique peuvent être autonomes et devenir une source de nouvelle propagation. Cela est illustré sur l'exemple du concept \og programme \fg{} en biologie (matériel génétique et sa fonction) qui a migré vers le domaine de l'informatique (opération d'un ordinateur). Les concepts sont donc capables de voyager d'une science à l'autre, ce qui a inspiré la métaphore des \og{}concepts nomades\fg{}, marqués par leur circulation spatio-temporelle et linguistique. Outre la nature itinérante des concepts scientifiques qui contribue à l'interdisciplinarité et à la production des savoirs nouveaux, \citeauthor{stengers1987d} (\citeyear{stengers1987d}, pp.~21-23) se réfère aux opérations de la \og capture \fg{} de la scientificité par ces concepts et du \og durcissement \fg{} conséquent des sciences. À savoir, certains concepts atteignent le degré de maturité après s'être avéré être adéquats et pertinents dans les démarches scientifiques dont ils \og{}capturent\fg{} la scientificité, permettant ainsi que le statut des sciences se solidifie ou \og durcisse \fg{}. La capture implique la définition, mais aussi la redéfinition d'une notion par les spécialistes d'une science. Les points de vue de \citeauthor{stengers1987d} (\citeyear{stengers1987d}) relèvent de la théorie constructiviste du savoir scientifique, selon laquelle la science est une \og{}construction\fg{} collective issue du contexte socio-historique (p. ex. interaction entre les scientifiques, les institutions etc.), et non pas d'une accumulation neutre et objective de faits.

Cette approche est complémentaire à l'histoire des concepts (allem. \textit{Begriffsgeschichte}), dans laquelle les significations des concepts en général sont considérées d'être les dérivés d'un contexte sociopolitique. Plus précisément, cette transformation d'un ou plusieurs mots en un concept survient lorsque cette construction linguistique comprend toute la gamme des significations dérivées d'un tel contexte \citep[p.~19]{koselleck2011introduction}. À titre d'exemple, le concept d'un \textit{état} ne peut être interprété qu'à travers ses différents constituants, dont \textit{souveraineté territoriale, législation, fiscalité}, parmi maints d'autres. L'histoire des concepts concerne principalement les manifestations de conflits sociopolitiques particuliers qui doivent être compris dans leur contexte approprié, où p. ex. les mots comme \textit{liberté} ou \textit{démocratie} portent la connotation polémique dont le sens ne peut être précisé qu'à travers leurs antithèses (\textit{esclavage} et \textit{dictature}, respectivement). Les concepts sont donc les concentrations par défaut ambiguës d'une multitude de contenus sémantiques, uniquement interprétables et indéfinissables, par contraste avec des significations des mots qui peuvent être définies de manière exacte \citep[p. 20]{koselleck2011introduction}. De plus, les concepts comme \textit{histoire} ou \textit{progrès} sont caractérisés comme \og{}collectifs singuliers\fg{} qui marquent un passage du domain concret d'un individu (plusieurs \textit{histoires} et \textit{progrès} individuels) au domain abstrait et général du collectif social (une \textit{histoire} ou un \textit{progrès} général ou collectif). Ce phénomène linguistique, ainsi que la création des concepts comme \textit{industrie, usine, classe moyenne} etc., reflète un changement de paradigme dans l'organisation sociale survenu lors des révolutions politiques et industrielles \citep[p. 1]{hobsbawm2010age}. La période charnière concernée par ce phénomène est nommée \textit{Sattelzeit}\footnote{Trad. allem. \og{}époque de selle\fg{}.}, entre 1750 et 1850, durant laquelle les concepts historiques deviennent abstraits, singularisés, respatialisés et retemporalisés \citep[pp.~34-35]{koselleck2011introduction}. Cela traduit le lien fort entre l'histoire du langage et l'histoire des idées.

Ces considérations sont appliquables à d'autres \og{}concepts nomades\fg{} en sciences humaines et sociales (ci-après \textsc{SHS}), comme \textit{travail}, \textit{intelligencija}, \textit{Ancien Régime}, \textit{avant-garde}, \textit{Occident} etc. qui font partie du \textit{Dictionnaire des concepts nomades en sciences humaines} \citep{christin2011dictionnaire}. Plusieurs questionnements ont été soulevés par \citeauthor{ghermani2011} (\citeyear{ghermani2011}, p.~117) eu égard de leur émergence, notamment pour déterminer à quel moment un concept devient une entrée dans un dictionnaire des \textsc{SHS} : \og{}\textit{Pourquoi un concept fait-il son entrée dans un dictionnaire ? Au terme de quel processus ? À l'inverse, comment cette percée lexicale est-elle parfois impossible ou refusée ?}\fg{}. Contrairement aux processus de la propagation et de la capture qui permettaient à un concept d'obtenir le statut de scientificité, l'autrice souligne les pratiques scientifiques conduisant aux rétractations et aux masquages de sens des concepts en \textsc{SHS}, p. ex. dans le cas du terme \og{}confession [religieuse]\fg{}, dont le sens varie en fonction de l'historiographie dans laquelle il figure \citep[p.~117]{ghermani2011}. Enfin, \citeauthor{bal2002travelling} (\citeyear{bal2002travelling}, p.~34) va plus loin en excluant la \og diffusion \fg{} et en mettant en avant la \og propagation \fg{} comme le critère discriminatoire de la nature itinérante des concepts. 

Pour résumer la complexité de la définition des concepts du point de vue de leur histoire, nous citons ici \citeauthor{bal2002travelling} (\citeyear{bal2002travelling}, p.~51), selon laquelle les concepts sont :
\begin{itemize}
\item datés, et donc marqués par une évolution ;
\item les mots : archaïsmes et néologismes relevant des mécanismes étymologiques qui leur donnent une dimension philosophique ;
\item syntaxiques au sein d'une langue ;
\item en évolution constante ;
\item créés, et non pas donnés \textit{a priori}.
\end{itemize}
\medskip
Concernant plus précisément le concept scientifique, l'épistémologie en esquisse les traits suivants, comme souligné par \citeauthor{rumelhard1986} (\citeyear{rumelhard1986}) et cité dans \citeauthor{astolfi2008chapitre} (\citeyear{astolfi2008chapitre}, p.~25) :
\begin{itemize}
\item le concept scientifique possède une dénomination et une définition, avec le sens le plus univoque possible, \textit{a contrario} du concept linguistique, en principe équivoque et polysémique ;
\item fonction opératoire : le concept scientifique est un outil intellectuel, un instrument théorique permettant d'interpréter des phénomènes ;
\item fonction d'opérateur, caractérisé par son degré de formalisation et par les interconnexions avec les techniques scientifiques ;
\item une extension, une compréhension, un domaine et des limites de validités en lien étroit avec sa définition fixée ;
\item le concept scientifique peut être compris comme un n\oe{}ud dans un réseau de relations organisé, au sein duquel il dialogue avec d'autres concepts et théories scientifiques.
\end{itemize}

\section{Repérage des termes scientifiques dans un corpus numérique}

Si nous nous limitons aux théories abordées jusqu'à maintenant, nous pouvons considérer que les concepts médicaux de Charcot ont eu le rôle des vecteurs de la crise conceptuelle, ce qui représentait une forme de \textit{Sattelzeit} dans le domaine de la médecine. Autrement dit, ces concepts ont été détournés de leurs sens initiaux ayant une apparence formelle neutre (descriptions des pathologies), vers ceux exerçant un certain impact sur la communauté scientifique que nous souhaitons mesurer informatiquement. Néanmoins, l'analyse numérique des concepts n'est pas une tâche triviale non plus, car tous les logiciels ne traitent pas des textes de la même manière. D'après \citeauthor{silberztein2022linguistic} (\citeyear{silberztein2022linguistic}, p.~2), les logiciels comme \textsc{TXM}\footnote{\url{https://txm.gitpages.huma-num.fr/textometrie/}}, Sketch Engine\footnote{\url{https://www.sketchengine.eu/}} ou \textsc{IR}a\textsc{M}u\textsc{T}e\textsc{Q}\footnote{\url{http://www.iramuteq.org/}} traitent les documents comme des \textit{séquences de formes graphiques} (dans notre cas, les séquences \og{}hystérie\fg{} et \og{}arthrite déformante\fg{} seraient composées d'une et de deux formes graphiques, respectivement). Ces formes sont définies comme les séquences contiguës de caractères alphabétiques délimités par des non-lettres ou les délimiteurs, qui peuvent être considérées comme des informations potentiellement  pertinentes pour une étude. D'autres logiciels, comme \textsc{N}oo\textsc{J}\footnote{\url{https://nooj.univ-fcomte.fr/}}, peuvent traiter ces séquences comme les \textit{unités linguistiques atomiques}, quel que soit le nombre de formes graphiques \citep[pp.~2-3]{silberztein2022linguistic}. Ainsi, l'unité linguistique atomique \og{}hystérie\fg{} serait recensée dans un dictionnaire des entrées lexicales simples (\textsc{DELAS}), tandis que \og{}arthrite déformante\fg{} ferait partie du dictionnaire des entrées lexicales composées (\textsc{DELAC})\footnote{Ce principe est repris lors du développement du logiciel Unitex \url{https://unitexgramlab.org/fr}.}.

Afin d'extraire automatiquement les concepts scientifiques, nous les opérationnalisons comme des \textit{termes} scientifiques. On en trouve une analogie proche dans la distinction terminologique relevée par \citeauthor{saussure1915} (\citeyear{saussure1915}, pp.~74-75) entre un \textit{signifié} (p. ex. le concept d'un arbre dans notre système cognitif) et un \textit{signifiant} (mot, parole, pictograme désignant un arbre) qui consitue un \textit{signe} (référent, un arbre réel). Les termes sont des expressions textuelles qui désignent des concepts dans un domaine d'expertise spécifique. Par conséquent, la tâche d'extraction des concepts peut donc être formalisée comme un problème d'extraction de la terminologie (angl. \textit{automatic text extraction -- \textsc{ATE}}), dont les enjeux appartiennent au domaine de l'extraction d'information (angl. \textit{information retrieval}), et plus largement, à celui du \textsc{TAL}. L'\textsc{ATE} a pour objectif de faciliter l'identification manuelle des termes à partir de corpus spécifiques à un domaine en fournissant une liste de termes candidats \citep[p.~1]{tran2023recent}.
\begin{itemize}
\item TermoStat \citep{drouin2003}
\item extraction terminologique \texttt{TermSuite} \citep{cram2016terminology}
\item approche linguistique : analyse syntaxique, POS tagging qui a ses limitations
\item approche statistique, mesure de pertinence : \textit{termhood} et \textit{unithood} \citep[pp.~6-7]{kageura1996}
\item approche apprentissage machine / profond : \citep{tran2023recent}
\item Pour nous, concept scientifique est opérationnalisé comme un terme scientifique.
\end{itemize}



%Motasem :
%revenir sur les concepts de l'index
%comment identifier les concepts médicaux dans les textes ? 
%regarder les fréquences des concepts médicaux
%quelles formes ?
%comment les identifier dans un texte ?
%lire la proposition de Motasem sur les anecdotes et celle de Glenn sur les embeddings dynamiques
%\textsc{ATISHS} : outil pour les HN 
%faire les stats à partir de la liste des concepts -- index (cooccurrences, pattern matching etc.) : qu'est-ce qui marche et marche pas ?

\textbf{Comment définir les concepts scientifiques du point de vue du TAL / analyse du corpus ? concepts, termes et mots-clés}

%Le mot \og{}concept\fg{} est un terme générique qui renvoie à un grand nombre de théories provenant de divers domaines de pensée, sans qu'il en existe une qui soit exhaustive et universellement acceptée.



En revanche, selon les linguistes, un concept a une structure double, constituée du sens linguistique et culturel.
%dont linguistique (générale, cognitive, psycholinguistique, ethnolinguistique), philosophie, métaphysique ou mathématiques
Sa couche intérieure est constituée du noyau étymologique sur lequel repose ensuite la couche périphérique qui hérite les éléments formés par la culture, les traditions et les expériences humaines
%\foreignlanguage{russian}{(Степанов, \citeyear{stepanov2007}}). 
\footnote{En linguoculturologie, on retrouve le terme \og{}concept linguo-culturel\fg{} qui reflète cette nature double du concept.}. Il peut être exprimé par de différentes éléments du langage, soit : lexèmes, idiomes, collocations, phrases ou textes entiers \citep[p.~5]{nemickiene2011concept}. 

Dans le domaine du traitement automatique des langues (\textsc{TAL}), le terme \og concept \fg{} peut s'apparenter à celui des \og entités nommées \fg{}, comme en témoignent les recherches sur l'extraction automatique de la terminologie biomédicale (\citealp{jolly2024exploring,navarro2023clinical}). Un concept d'un domaine de connaissance peut faire partie d'un thésaurus, liste organisée de termes contrôlés et normalisés, auquel cas le concept est appelé \og descripteur \fg{}. \citep[p.~16]{RENNESSON202015}.

Un exemple de ce phénomène est le terme \textsc{mot}, qui véhicule une réalité particulière appartenant à chaque langue (\citeauthor{mounin1968clefs} \citeyear{mounin1968clefs}, p. 65). 

Nous n'entendons pas le terme \textsc{concept} dans le sens de Saussure,.... signe = concept (signifié) + image acoustique (signifiant)

Même si l'on reprend la description de Saussure qui considère le mot comme \og une image acoustique associé à un concept \fg{}, nous nous heurtons ensuite au problème de la définition du terme \textit{concept}. Le structuralisme linguistique de Bloomfield souligne ce point, en ajoutant que les linguistes ne sont pas outillés pour démêler complètement ce réseau complexe. Ce structuraliste poursuit en disant que le langage peut en effet être perçu comme une abstraction construite à partir de nos connaissances sur celui-ci, mais qu'il faut \og décrire d'abord le fonctionnement de cet instrument de communication \fg{} et expliquer comment nous (dé)construisons les énoncés en tant que locuteurs ou auditeurs (\citeauthor{mounin1968clefs} \citeyear{mounin1968clefs}, pp. 94-95).

\begin{itemize}
\item ok, et c'est quoi le concept en linguistique (de Saussure) et en analyse du discours
\item nous différencions des concepts des \og{}figements linguistiques\fg{} \citep{bezancon2023}
\end{itemize}



Dans le souci de différencier ces notions à travers les disciplines citées, nous présentons ci-dessous quelques-uns de leurs traits discriminatoires qui ne prétendent être ni exhaustifs ni limitatifs :

\begin{table}[h]
\centering
\begin{tabular}{|l|l|l|l|}
\hline
        & \multicolumn{1}{c|}{\begin{tabular}[c]{@{}c@{}}Philosophie\\ Épistémologie\end{tabular}} & \multicolumn{1}{c|}{Linguistique} & \multicolumn{1}{c|}{TAL} \\ \hline
\textsc{Idée}    & objet de connaissance                                                                                          \citep[p.~261]{Lecourt1999} &                                   &                          \\ \hline
\textsc{Concept} & représentation de l'objet de connaissance \citep[p.~261]{Lecourt1999}                                                                                      \\ \hline 
\textsc{Signifié} &  \citep[p.~27]{astolfi2008chapitre}                                                                                      \\ \hline
\textsc{Signifiant} &    mode de représentation des signifiés \citep[p.~27]{astolfi2008chapitre}                              &                          \\ \hline
\textsc{Terme}   &                                                                                          &                                   &                          \\ \hline
\textsc{Mot} &                                                                                          &                                   &                          \\ \hline
\textsc{Mot-clé} &                                                                                          &                                   &                          \\ \hline
\end{tabular}
\end{table}

\begin{itemize}
\item \textbf{Proposer de formaliser la définition du concept (identifiables dans un corpus), mots clés ? Embeddings ? —>} 
\item nous nous appuyons sur une approximation d'un tel concept, car la tâche d'automatisation et d'implémentation dans l'optique computationnelle enlève forcément quelques traits de concepts abordés dans ce chapitre
\end{itemize}





\section{Études numériques des circulations culturelles}
\label{circulations}
Incontestablement, l'époque actuelle est profondément marquée par le \og{}déluge des données\fg{}, phénomène représentatif de la quatrième paradigme de la science, selon Jim Gray \citep[p.~30]{hey2009jim}. Par conséquent, les projets numériques sont aujourd'hui \og{}pilotés par les données\fg{}\footnote{Traduction du terme \textit{data-driven} introduit par \citet{Johns1991ShouldYB}, issu de l'expression \textit{data-driven learning}.} et ceux qui sont centrés sur les explorations des circulations culturelles au prisme du numérique se concrétisent à grande échelle.
%Les humanités numériques au service de l'analyse des circulations culturelles se manifestent sous forme de divers projets de recherche au niveau académique. 
Sont fortement axés sur cette thématique :
\begin{enumerate}
\item certaines chaires universitaires, notamment celle des Humanités numériques à l'université de Genève \citep{joyeux2022circulations}\footnote{\textit{Cf.} les projets de la chaire : \url{https://www.unige.ch/lettres/humanites-numeriques/recherche/projets-de-la-chaire}.} ;
\item de divers évènements scientifiques, comme la journée d'étude \og{}Circulation des écrits littéraires de la première modernité et humanités numériques\fg{}\footnote{\url{https://www.fabula.org/actualites/86846/circulation-des-ecrits-litteraires-de-la-premiere-modernite-et-humanites-numeriques.html}}, les colloques Humanistica 2023\footnote{\url{https://humanistica2023.sciencesconf.org/}}, \textsc{ACFAS} 2023\footnote{\url{https://www.crihn.org/nouvelles/2022/12/11/colloque-de-la-transformation-des-sciences-humaines-par-les-humanites-numeriques-acfas-2023/}} etc. ;
\item des numéros de certaines revues, par exemple \og{}Circulation des discours dans les récits complotistes\fg{}, dont les articles portent sur les thématiques aussi diverses que les circulations textuelles internationales du discours complotiste des \og Illuminati \fg{}  \citep{chaudet2022illuminati}, \og conspirationniste \fg{} sur Twitter \citep{giry2022etudier} ou antiféministe en ligne \citep{morin2022discours}. 
\end{enumerate}

%Ce mémoire est basé sur la contribution de \citet{petkovic2023circulation} s'inscrivant dans l'optique de l'exploration des circulations médicales. Nous souhaitons mesurer informatiquement l'impact scientifique des travaux de Charcot sur ses collaborateurs et successeurs, membres de son réseau scientifique. Cette mesure se fonde sur l'analyse des concepts-clés en matière de son discours scientifique, et plus particulièrement sur l'opérationnalisation du terme \og{}influence\fg{}, définie ici comme une intertextualité\footnote{Nous nous appuyons sur la définition de l'intertextualité dans la littérature, où ce terme désigne \og{}la perception, par le lecteur, de rapports entre une \oe{}uvre et d'autres, qui l'ont précédée ou suivie\fg{} \citep[p.~4]{riffaterre1980trace}.} uni-directionnelle, allant des écrits de Charcot (ci-après corpus \og{}Charcot\fg{}) vers ceux de ses collaborateurs et successeurs (ci-après corpus \og{}Autres\fg{}). Il s'agit donc \textit{in fine} d'aborder computationnellement la question des circulations, non pas des artefacts matériels comme les manuscrits \citep{gabay2021katabase} et les images \citep{joyeux2019visual}, mais des phénomènes textuels complexes \citep{manjavacas} ayant une dimension théorique forte. 

La question de recherche sous-tendant ce mémoire s'approche tangentiellement des travaux de \citet{riguet2018impact} et de \citet{roe2023enlightenment}. Le premier travail porte sur la réception de la pensée scientifique du physiologiste français Claude Bernard dans la critique littéraire, illustrée par l'alignement des textes de Bernard avec des ouvrages de critique littéraire. Le second article porte sur la détection de réemplois textuels à grande échelle et l'analyse de réseaux pour identifier les \og{}influenceurs\fg{} dans les ouvrages français du siècle des Lumières.

Pour ce qui est des projets individuels, la question de l'estimation de l'importance d'une entité issue d'un domaine ontologique occupe une place centrale dans le travail de \citet{soulet2024}, ce qui a résulté dans le développement de l'outil de représentation des connaissances Rankingdom\footnote{\url{https://rankingdom.org/about}.}. L'un de ses aspects concerne les déclinaisons de la notion d'importance d'une entité résultant aux métriques correspondantes, comme présenté dans le tableau \ref{tab:rankingdom} : ces métriques sont calculées pour l'entité \texttt{Jean-Martin Charcot}.

\begin{table}[H]
\centering
\footnotesize
\resizebox{\textwidth}{!}{ % Resize the table to fit within text width
\begin{tabular}{|c|c|c|}
\hline
 \rowcolor{maroon!10}
\begin{tabular}[c]{@{}c@{}}\textit{Métrique}\end{tabular} & \begin{tabular}[c]{@{}c@{}}\textit{Définition}\end{tabular}                                                                                         & \begin{tabular}[c]{@{}c@{}}\textit{Exemple}\end{tabular}                                                                                                                                             \\
\hline
\begin{tabular}[c]{@{}c@{}}\textsc{Portée}\\ (\textsc{Popularité})\end{tabular} & \begin{tabular}[c]{@{}l@{}}nombre d'assertions décrivant une entité\end{tabular}                                                                                         & Charcot est décrit par 546 assertions.                                                                                                                                                \\
\hline
\textsc{Influence}                                                     & \begin{tabular}[c]{@{}l@{}}nombre d'entités liées à une entité\end{tabular}                                                                                        & 191 entités liées à Charcot.                                                                                                                                                     \\
\hline
\textsc{À propos}                                                      & \begin{tabular}[c]{@{}l@{}}nombre d'entités impactées \\ (\oe{}uvres originales, évènements$\dots$)\end{tabular}                                                                    & 56 entités à propos de Charcot.                                                                                                                                                  \\
\hline
\textsc{Index \textit{a}}                                                       & \begin{tabular}[c]{@{}c@{}}nombre maximum d'entités impactées \textit{a} \\
ayant le comptage \og{} à propos \fg{}\end{tabular} & \begin{tabular}[c]{@{}c@{}}13 entités impactées par Charcot ont \\ le comptage \og{} à propos \fg{} supérieur à 13.\end{tabular} \\
\hline
\textsc{Impact}                                                        & \begin{tabular}[c]{@{}c@{}}somme de tous les comptages \og{} à propos \fg{}\\
de toutes les entités impactées\end{tabular}               & L'impact de Charcot est 826.                                                                                                                                                         \\
\hline
\end{tabular}
}
\caption{Aperçu des métriques Rankingdom pour quantifier l'importance de l'entité \texttt{Jean-Martin Charcot}.}
\label{tab:rankingdom}
\end{table}

De plus, des calculs effectués à partir de la portée et de l'influence de Charcot permettent de générer un graphique de \og{}quadrant magique de Gartner\fg{} (angl. \textit{Gartner Magic Quadrant})\footnote{Le nom provient de la société américaine de conseil Gartner qui \og{}publie chaque année les résultats de ses analyses dans plus de 100 secteurs technologiques\fg{} \citep{gartner}.}. Cette représentation sur la figure \ref{fig:analyse_quadrant} met en valeur quatre types d'entités : 
\begin{itemize}
\item \textbf{acteurs de niche} : entités avec une portée et une influence modestes (p. ex. Pierre Marie) ;
\item \textit{\textbf{challengers}} : entités ayant une certaine reconnaissance et une influence considérable, mais qui sont de taille mineure, more concentrées, avec une portée plus petite (Charles-Joseph Bouchard) ;
\item \textbf{visionnaires} : entités avec une grande portée, dont l'influence reste néanmoins limitée et qui recevront plus de reconnaissance ultérieurement (Paul Richer) ;
\item \textit{\textbf{leaders}} : entités les plus importantes, avec une grande portée, connues à grande échelle et dans plusieurs domaines, tout en étant reconnues comme ayant une grande influence (Charcot).
\end{itemize}

\begin{figure}[H]
    \centering
    \includegraphics[width=1\textwidth]{img/analyse_quadrant.png}
    \caption[Positionnement de l'entité \texttt{Jean-Martin Charcot} au sein de son domaine et comparaison avec les entités les plus similaires à lui \textit{via} une analyse de quadrant de l'outil Rankingdom.]{Positionnement de l'entité \texttt{Jean-Martin Charcot} au sein de son domaine et comparaison avec les entités les plus similaires à lui \textit{via} une analyse de quadrant de l'outil Rankingdom\protect\footnotemark{.}}
    % Pour raison de visibilité, l'image originale a été agrandie, ce qui a entraîné le rapprochement des années sur l'axe de l'abscisse.
    \label{fig:analyse_quadrant}
\end{figure}

\footnotetext{\url{https://www.rankingdom.org/entity/Q20710?search=jean-martin+charcot}. Le domaine dans lequel Charcot figure est relativement large, y compris les figures du domaine médical (p. ex. Bourneville), mais aussi littéraire (Arsène Arnaud Claretie).}

\bigskip
L'impact de Charcot peut également être visualisé à l'aide de Rankingdom à travers le graphique qui apporte une dimension temporelle (figure \ref{fig:impact_temporel}). Il s'agit notamment de la cumulation temporelle de son impact, où l'on peut observer qu'il s'étend sur la période 1856--1994. 

\begin{figure}[hb]
    \centering
    \includegraphics[width=1\textwidth]{img/impact_temporel.png}
    \caption[Analyse temporelle de l'impact de l'entité \texttt{Jean-Martin Charcot} à l'aide de l'outil Rankingdom.]{Analyse temporelle de l'impact de l'entité \texttt{Jean-Martin Charcot} à l'aide de l'outil Rankingdom\protect\footnotemark{.}}
    % Pour raison de visibilité, l'image originale a été agrandie, ce qui a entraîné le rapprochement des années sur l'axe de l'abscisse.
    \label{fig:impact_temporel}
\end{figure}

\footnotetext{\url{https://www.rankingdom.org/entity/Q20710?search=jean-martin+charcot}.}

Enfin, il est également possible de lister les entités impactées, comprenant les personnes (p. ex. Sigmund Freud), les notions médicales (SEP) ou bien les entités géographiques afférentes (Île Charcot).


%Les humanités numériques au service de l'analyse des circulations culturelles
%
%Comment définir une circulation du point de vue de l'analyse du texte ? de la linguistique computationnelle (TAL) ?


\chapter{Méthodologie}
\label{methodo}
\section{Constitution du corpus Charcot}

\section{Outils existants}
\section{Une nouvelle approche}



\chapter{Résultats}
\label{resultats}
%\minitoc
\section{Exploration du corpus Charcot : \textsc{OBVIE} et \textsc{TextPair}}
Une première exploration du corpus Charcot à travers l'application OBVIE nous a permis d'identifier les substantifs les plus importants de chaque corpus en utilisant les fréquences brutes ou des méthodes plus fines comme \textsc{TF-IDF}, \textsc{BM25} (détaillées dans la partie \ref{methodo_stat}), \textsc{$\chi$2} ou le \textsc{Test Gamma}. Cependant, l'application ne permet pas de quantifier la pertinence des expressions polylexicales, soit les n-grammes de mots, très fréquentes dans les deux corpus et dont la décomposition entraînerait une perte d'information (p. ex. le terme polysémique \og{}bulbe\fg{} qui a une valeur spécifique dans l'expression figée \textit{bulbe rachidien}). En observant la figure \ref{fig:bulbe}, nous constatons que l'abscisse donne l'information sur les dates de publication des ouvrages compris dans les corpus, alors que l'ordonnée indique le nombre d'occurrences par million de mots, soit \textit{parties par million} (\textit{ppm})\footnote{\textit{Cf.} le guide d'utilisation d'\textsc{OBVIE} détaillé : \url{https://obtic.huma-num.fr/obvie//static/aide.html}.}. 
\begin{figure}[!ht]
    \centering
    \includegraphics[width=1\textwidth]{img/bulbe_rachidien_mini.png}
    \caption{Distribution des fréquences des tokens avec la frise chronologique pour ceux constituant l'expression \og{}bulbe rachidien\fg{} (issus du corpus \og{}Charcot\fg{} et du corpus \og{}Autres\fg{}) dans le logiciel OBVIE.
    % Pour raison de visibilité, l'image originale a été agrandie, ce qui a entraîné le rapprochement des années sur l'axe de l'abscisse.
    }
    \label{fig:bulbe}
\end{figure}

Concernant l'alignement des séquences similaires aux deux corpus, \textsc{TextPair} nous a permis, par une lecture attentive, de faire des comparaisons entre les textes et de rechercher des termes au sein des passages similaires, malgré le nombre de résultats assez conséquent (\textit{cf}. la figure \ref{fig:textpair}). En raison de sa capacité de détecter les passages similaires, notamment les citations directes, les plagiats ou les réemplois, ce logiciel, ainsi qu'un autre logiciel de détection de plagiat, peuvent nous servir de \textit{baseline} pour comparer leurs résultats avec ceux proposés dans la partie \ref{methodo_stat}.

\begin{figure}[!ht]
    \centering
    \includegraphics[width=1\textwidth]{img/textpair.png}
    \caption{Alignement et comparaison d'un texte de Charcot à celui de Georges Gilles de la Tourette (le seul résultat) en lançant la requête \textit{sclérose latérale amyotrophique}.}
    \label{fig:textpair}
\end{figure}

\section{Extraction de la terminologie : approche linguistique}

Dans le cadre de l'approche linguistique de l'extraction terminologique, nous avons tenté d'utiliser l'outil \texttt{TermoStat}. Bien que le traitement d'un échantillon minuscule des corpus (1-2 documents) ait pu se terminer avec succès, le passage à l'échelle de l'intégralité des corpus n'a pas été possible en raison des limitations citées dans la section \ref{termes}. Même en respectant la limite du corpus de 30 Mo, le traitement a été extrêmement chronophage, sans pour autant générer aucun résultat après plusieurs jours de calcul. En revanche, l'utilisation de l'outil \texttt{TermSuite} s'est avéré comme un moyen alternatif bien plus efficace pour générer les résultats souhaités, étant donné que le traitement de chaque corpus a duré environ une vingtaine de minutes\footnote{Les résultats sont disponibles dans le dépôt GitHub \url{https://github.com/ljpetkovic/Charcot\_TermSuite}}. Nous disposons de deux tableaux issus de l'extraction des termes uniques correspondant aux deux corpus, ainsi que de leurs diverses caractéristiques et mesures statistiques (motifs syntaxiques des parties du discours, fréquences brute et documentaire, \textsc{TF-IDF}, spécificité$\dots$). Concernant les motifs syntaxiques, nous en avons extrait 6 types, marqués par leurs étiquettes \texttt{TreeTagger} comme illustré dans le tableau \ref{tab:POS_tags}. 
\begin{table}[h]
	\centering
	\begin{tabular}{c|c}
		Étiquette & Signification \\\hline
		\textsc{A} & \texttt{adjectif}\\
		\textsc{N} & \texttt{nom}\\ 
		\textsc{N A} & \texttt{nom + adjectif}\\
		\textsc{N A A} & \texttt{nom + adjectif + adjectif}\\
		\textsc{N P N} & \texttt{nom + préposition + nom}\\
		\textsc{R} & \texttt{adverbe}
	\end{tabular}
	\caption{Étiquettes \texttt{TreeTagger} extraites avec \texttt{TermSuite}, accompagnées de leurs significations.}
	\label{tab:POS_tags}
\end{table} 

Le tableau \ref{tab:repartition_POS} montre les motifs syntaxiques extraits, avec leurs fréquences absolues (nombres d'occurrences extraites), leurs fréquences relatives (pourcentages de toutes les étiquettes extraites), avec des exemples des termes représentatifs correspondant à chaque motif.
\begin{table}[h]
	\centering
	\resizebox{\textwidth}{!}{
		\begin{tabular}{|lrrr|rrl|}
			\hline\hline
			\rowcolor{gray!10}\multicolumn{4}{|c|}{Corpus Charcot}                                                                    & \multicolumn{3}{|c|}{Corpus Autres}                               \\ \hline
			\multicolumn{1}{|c|}{Motif POS}      & \multicolumn{1}{c|}{Effectif} & \multicolumn{1}{c|}{Fréq. relat. (\%)} & \multicolumn{1}{c|}{Exemple} & \multicolumn{1}{c|}{Effectif} & \multicolumn{1}{c|}{Fréq. relat. (\%)} & \multicolumn{1}{c|}{Exemple}\\ \hline
			\rowcolor{yellow!30}\multicolumn{1}{|c|}{\textsc{N}}              & \multicolumn{1}{r|}{261}               & \multicolumn{1}{r|}{52,10}                   & \multicolumn{1}{c|}{\textit{hystérie}} & \multicolumn{1}{r|}{271}               & \multicolumn{1}{r|}{54,20} & \multicolumn{1}{c|}{\textit{somnambule}}                  \\ \hline
			\multicolumn{1}{|c|}{\textsc{A}}              & \multicolumn{1}{r|}{151}               &  \multicolumn{1}{r|}{30,14}                   & \multicolumn{1}{c|}{\textit{cérébral}} & \multicolumn{1}{r|}{149}               & \multicolumn{1}{r|}{29,80}      &   \multicolumn{1}{c|}{\textit{hypnotique}}          \\ \hline
			\multicolumn{1}{|c|}{\textsc{N A}}            & \multicolumn{1}{r|}{73}                & \multicolumn{1}{r|}{14,57}                   &
			\multicolumn{1}{c|}{\textit{système nerveux}} 	& \multicolumn{1}{r|}{73}                & \multicolumn{1}{r|}{14,60}       	&
			\multicolumn{1}{c|}{\textit{lame médullaire}}            \\ \hline
			\multicolumn{1}{|c|}{\textsc{N P N}}          & \multicolumn{1}{r|}{12}                & \multicolumn{1}{r|}{2,40}                    &
			\multicolumn{1}{c|}{\textit{cas de folie}} &
			\multicolumn{1}{r|}{6}                 & \multicolumn{1}{r|}{1,20}   &
			\multicolumn{1}{c|}{\textit{scissure de sylvius}}                 \\ \hline
			\multicolumn{1}{|c|}{\textsc{N A A}}          & \multicolumn{1}{r|}{3}                 & \multicolumn{1}{r|}{0,60}                    &
			\multicolumn{1}{c|}{\textit{système nerveux central}} &
			\multicolumn{1}{r|}{0}                 & \multicolumn{1}{r|}{0,00}           &
			\multicolumn{1}{c|}{--}         \\ \hline
			\multicolumn{1}{|c|}{\textsc{R}}              & \multicolumn{1}{r|}{1}                 & \multicolumn{1}{r|}{0,20}                    &
			\multicolumn{1}{c|}{\textit{[d']emblée}} &
			\multicolumn{1}{r|}{1}                 & \multicolumn{1}{r|}{0,20}       &
			\multicolumn{1}{c|}{\textit{obliquement}}             \\ \hline\hline
			\multicolumn{1}{|c|}{\textbf{Total}} & \multicolumn{1}{r|}{\textbf{501}}      & \multicolumn{1}{r|}{100,00}                  &
			\multicolumn{1}{r|}{\cellcolor{blue!25}} &
			\multicolumn{1}{r|}{\textbf{500}}      & \multicolumn{1}{r|}{100,00} &
			\multicolumn{1}{r|}{\cellcolor{blue!25}}                 \\ \hline\hline
		\end{tabular}
	}
	\caption{Répartition des parties du discours constituant les termes médicaux dans les corpus \og{}Charcot\fg{} et \og{}Autres\fg{}.}
	\label{tab:repartition_POS}
\end{table}
Nous soulignons que 5 motifs sont communs aux deux corpus, hormis celui de \texttt{[N A A]}, extrait uniquement à partir du corpus \og{}Charcot\fg{}. Toutefois, il est intéressant de noter qu'aucune occurrence du terme très fréquent \textit{sclérose latérale amyotrophique}, ainsi que de ses éléments constitutifs \textit{sclérose} et \textit{amyotrophique}, n'a été extraite ; on n'en retrouve les traces que dans l'adjectif \texttt{[A]} extrait \textit{latérale}. Les trigrammes sont les séquences les plus longues extraites, notamment 6 occurrences de \texttt{[N P N]} (corpus \og{}Autres\fg{}), 12 de \texttt{[N P N]} et 3 de \texttt{[N A A]} (les deux dernières dans le corpus \og{}Charcot\fg{}). Les séquences plus longues sont très pertinentes pour l'extraction de la terminologie précise (p. ex. \textit{sclérose} est terme moins précis que \textit{sclérose latérale amyotrophique}). Dans la figure \ref{fig:repartition_POS}, nous exposons les répartitions des motifs syntaxiques constituant les termes médicaux extraits en termes de leurs effectifs dans les corpus \og{}Charcot\fg{} et \og{}Autres\fg{}. Nous nous apercevons que les motifs les plus fréquents sont les unigrammes (mots individuels) de noms \texttt{[N]} et les adjectifs \texttt{[A]}, alors que les bigrammes et les trigrammes sont présents dans une moindre mesure. Cela laisse à penser que \texttt{TermSuite} ne parvient pas à extraire les termes médicaux sous forme de quadrigrammes (séquence de quatre mots consécutifs, p. ex. \textit{sclérose en plaques disséminées}) ou des séquences plus longues (\textit{état de mal hystéro-épileptique}) qui sont bel et bien mentionnées dans les deux corpus. En plus, cette expérience confirme les limites de l'approche linguistique de l'extraction des termes scientifiques à base de règles, notamment à l'aide des expressions régulières et des automates à états finis. En effet, il est bien connu que leur construction est une tâche fastidieuse, restrictive et non maintenable sur le long terme, surtout en cas d'un grand ensemble de termes.

\begin{figure}[h] % Use [H] to force the figure to stay in place
	\centering
	\includegraphics[width=\linewidth]{img/repartition_motifs_POS.png}
	\caption{Analyse comparative des séquences syntaxiques constituant les termes scientifiques dans le corpus \og{}Charcot\fg{} et \og{}Autres\fg{}.}
	\label{fig:repartition_POS}
\end{figure}

Toutefois, \texttt{TermSuite} donne la possibilité d'analyser la pertinence des termes à travers les mesures statistiques. Pour ce qui est de celle de \textsc{TF-IDF},
%une corrélation linéaire positive forte avec le coefficient de détermination \textsc{R}\textsuperscript{2} de $0,93$ (situé dans l'intervalle $[-1, 1]$) a été constatée entre la fréquence brute et la mesure \textsc{TF-IDF} des termes dans le corpus \og{}Charcot\fg{} (indiquée en rouge sur la figure \ref{fig:correlation_Charcot}). 
%Autrement dit, plus un terme est fréquent, plus il est considéré comme pertinent selon la mesure \textsc{TF-IDF}. 
elle mesure la pertinence d'un terme en calculant sa fréquence brute moins l'inverse de sa fréquence documentaire (le nombre de documents contenant ce terme indique sa dispersion). L'intuition derrière cette mesure se résumerait ainsi : si un terme apparaît souvent dans quelques documents, il s'agit d'un terme spécialisé. En revanche, si un terme apparaît souvent dans beaucoup de documents, il sera moins pertinent (un exemple classique de ce phénomène sont les mots vides, p. ex. \textit{par exemple}). 
%La moyenne indiquée par la ligne verte représente la répartition des valeurs \textsc{TF-IDF} normalisées entre $0$ et $1$ autour de leur valeur centrale (\textsc{0,09}), tandis que la médiane (\textsc{0,04}), indiqué en bleu clair, sépare les données en deux groupes égaux. 
%Les valeurs \textsc{TF-IDF} sont inférieures dans le cas des bigrammes et des trigrammes dans le corpus \og{}Charcot\fg{}, contrairement aux unigrammes qui sont classés comme plus pertinents. Par exemple, \textit{pathologie nerveuse} et \textit{cas de paralysie} sont le bigramme et le trigramme les mieux classés avec leurs valeurs de \textsc{TF-IDF} de $0,33$ et de $0,05$, respectivement.
Nous remarquons que les bigrammes comme, p. ex. \textit{paralysie générale} ($0,53$) et \textit{lame médullaire} ($0,51$) n'ont été sous-valorisés dans aucun corpus par rapport aux unigrammes selon cette mesure, puisqu'ils figurent dans les parties supérieures des listes des termes extraites. Cela ne s'applique pas aux trigrammes comme \textit{troubles de sensibilité} ($0,10$) et \textit{pli de passage} ($0,14$). Enfin, nous avons également récupéré les résultats de la spécificité des termes extraits. Cette mesure se réfère au ratio d'étrangeté (angl. \textit{weirdness ratio}) \citep{khurshid2000weirdness}, soit à la \og{}termicité\fg{} des termes dans le corpus par rapport au langage général. Elle exprime le degré de relation d'un terme avec un domaine spécifique. Par exemple, le terme \textit{atrophie} a une termicité plus élevée ($4,44$) que \textit{maladie nerveuses} ($3,15$) dans le corpus \og{}Charcot\fg{} ; le même rapport est observé pour les termes \textit{atrophie} ($4,38$) et \textit{antécédent personnel} ($3,17$) dans le corpus \og{}Autres\fg{}.

%\begin{figure}[h] % Use [H] to force the figure to stay in place
%	\centering
%	\includegraphics[width=\linewidth]{img/correlation_Charcot.png}
%	\caption{Corrélation linéaire positive entre la fréquence brute et \textsc{TF-IDF} dans le corpus \og{}Charcot\fg{}.}
%	\label{fig:correlation_Charcot}
%\end{figure}

%\begin{figure}[h] % Use [H] to force the figure to stay in place
%	\centering
%	\includegraphics[width=\linewidth]{img/correlation_Autres.png}
%	\caption{Corrélation linéaire positive entre la fréquence brute et \textsc{TF-IDF} dans le corpus \og{}Charcot\fg{}.}
%	\label{fig:correlation_Autres}
%\end{figure}











\section{Extraction des phrases-clés : méthodes statistiques}
\label{methodo_stat}
Afin de surmonter les limites rencontrées avec ces deux outils, nous avons proposé une nouvelle méthode pour identifier des concepts dans les deux corpus en nous basant sur le poids de leur apparition, calculé selon trois différentes mesures de pondération\footnote{Le code est disponible en ligne : \url{https://github.com/ljpetkovic/Charcot\_circulations}.} :
\begin{itemize}
\item \textsc{TF-IDF} \citep{robertson1976relevance} est une méthode qui permet d'évaluer l'importance d'un terme contenu dans un document relativement à un corpus plus large en récompensant la fréquence des termes, sans tenir compte des variations de longueur du document ;
\item \textsc{BM25} est une fonction de classement qui classe un ensemble de documents en fonction des termes de requête apparaissant dans chaque document, quelle que soit l'interrelation entre les termes de requête au sein d'un document (par exemple, leur proximité relative). Il s'agit d'une tentative d'amélioration de \textsc{TF-IDF}, notamment pour prendre en compte divers facteurs tels que la longueur du document et les problèmes engendrés par la possible saturation des termes \citep[p.~355]{robertson2009probabilistic} ;
\item \textsc{BERT} \citep{devlin2019} est un modèle pré-entraîné qui utilise l'apprentissage non-supervisé sur de grandes quantités de données textuelles pour apprendre des représentations de mots et de phrases, et comprendre le contexte et la sémantique. Il est basé sur l'architecture des \textit{transformeurs}, qui est un type de grand modèle de langue utilisé pour le \textsc{TAL}.
\end{itemize}

La liste des concepts retenus pour l'étude est composée de termes ou expressions popularisés par Charcot, comme \textit{hystérie}, \textit{sclérose latérale} etc. \citep[p.~1102]{camargo2024} \footnote{\textit{Cf.} la liste exhaustive des termes et des expressions popularisés par Charcot en annexe.}. Pour chaque entrée, nous avons pris en compte les formes du singulier et du pluriel obtenues grâce à des expressions régulières. La liste est  produite de façon supervisée et provient du croisement entre la liste des termes obtenus avec OBVIE et l'index d'une édition des \oe{}uvres complètes de \cite[pp.~493--507]{charcot1892oeuvres}, dont nous avons retiré les termes génériques (\textit{os}, \textit{cerveau}, etc.).

Comme nous pouvons l'observer sur la figure \ref{fig:bm25}, la mesure \textsc{BM25} révèle une intensification du lexique de Charcot dans le corpus \og{}Autres\fg{}. Plus précisément, tous les termes évalués sont identifiés comme plus signifiants dans le discours des \og{}Autres\fg{} que dans celui de Charcot, les scores étant plus élevés pour 14 termes (sur 14 évalués) utilisés par le réseau de Charcot. D'ailleurs, d'après le tableau \ref{tab:calculs_stat} (en annexe), c'est la seule mesure dont les valeurs témoignent clairement d'un lexique partagé entre Charcot et ses successeurs et collaborateurs, \textit{a contrario} des deux autres mesures, où le rapport en question est inversé (la grande majorité des termes étant plus pertinente dans le discours de Charcot, et son impact étant donc moins accentué). Concrètement, les termes les plus pertinents semblent être \textit{sclérose en plaque disséminées} (score 0,83), \textit{paralysie rhumatismale} (0,68), \textit{atrophie progressive} (0,53) et \textit{arthrite déformante} (0,50).

\begin{figure}[!h]
    \centering
    \includegraphics[width=1\textwidth]{img/Charcot_Autres_250523.png}
    \caption{Visualisation de pertinence des concepts dans les deux corpus suivant la métrique \textsc{BM25}. Les valeurs des concepts associées au corpus \og{}Autres\fg{} sont représentées en bleu, alors que celles du corpus \og{}Charcot\fg{} en jaune.}
    \label{fig:bm25}
\end{figure}

D'autre part, nous avons utilisé \textsc{BERT} pour mesurer le poids des termes dans les deux corpus. Bien que ce type de modèle ne fournisse pas directement de poids pour les mots, nous pourrions cependant en extraire des informations utiles pour estimer l'importance ou le poids des mots dans les textes. Différentes approches sont généralement utilisées pour obtenir une représentation de l'importance des mots, en exploitant les informations des plongements lexicaux et des mécanismes d'attention \citep{vaswani2023}. Pour ce travail en cours, nous avons utilisé le modèle \texttt{bert-base-multilingual-cased}. Les premiers résultats obtenus se trouvent dans le tableau \ref{tab:calculs_stat} et restent à améliorer. Cependant, nous avons observé que les termes les plus pertinents pour le discours de Charcot étaient ceux qui désignent les noms des différentes pathologies (\textit{diplopie}, \textit{myélite partielle}, \textit{état de mal épileptique}, \textit{paralysie labio-glosso-laryngée} etc.), contrairement à d'autres notions plus abstraites (\textit{vicieuses}, \textit{délire}, \textit{miracle}) qui sont prédominantes dans le corpus
\og{}Autres\fg{} (termes non renseignés dans le tableau en question). La présence de ce dernier type de notion n'est pas étonnant, étant donné que Charcot aborde la question des guérisons miraculeuses dans ses recherches\footnote{Voir notamment son \oe{}uvre \textit{La foi qui guérit} \citep{charcot1897foi}.}. 



\section{Extraction des phrases-clés : méthode à base d'apprentissage profond}
En complément de la méthode du calcul de pertinence des termes médicaux fournis de manière supervisée (partie \ref{methodo_stat}), nous exposons ici des résultats de l'approche non-supervisée pour extraire des mots/phrases-clés  pertinents à partir de nos deux corpus\footnote{\textit{Cf.} le dépôt GitHub \url{https://github.com/ljpetkovic/Charcot\_KeyBERT\_Keyphrase-Vectorizers/}.}. L'objectif de cette approche est de détecter les termes communs entre les deux corpus et de montrer la répartition des termes les plus pertinents dans le réseau de Charcot. Deux algorithmes librement disponibles sont présentés ici pour illustrer cette dernière approche : \texttt{keybert} \citep{grootendorst2023}\footnote{\url{https://maartengr.github.io/KeyBERT/}} et \texttt{keyphrase-vectorizers}\footnote{\url{https://pypi.org/project/keyphrase-vectorizers/}}. Lors du passage à l'échelle avec la quantité de données considérable (voir le tableau \ref{tab:corpus}), nous avons fait face un manque de puissance de calcul des ordinateurs locaux. Pour faciliter l'extraction des phrases-clés, nous avons obtenu l'accès à la plateforme technologique \textsc{MeSU} et un accompagnement technique grâce à l'unité de service \textsc{SACADO} (Service d'Aide au Calcul et à l'Analyse de Données)\footnote{\url{https://sacado.sorbonne-universite.fr/fr/}.} de Sorbonne Université.

\subsection{Librairie \texttt{keybert}}

Cette librairie Python permet d'exploiter les plongements de mots (angl. \textit{word embeddings}) du type \textsc{BERT} pour générer des mots/phrases-clés les plus similaires à un document.
La figure \ref{fig:keybert} illustre la chaîne de traitement appliquée à nos deux corpus : 
\begin{enumerate}
\item les corpus \og Charcot \fg{} et \og Autres \fg{} sont utilisés comme les données d'entrée au format \texttt{.txt} ;
\item les documents d'entrée ont été tokenisés en phrases-clés candidates avec la fonction \texttt{CountVectorizer} ; 
\item les plongements des documents et de leurs phrases-clés candidates ont été générés par le modèle de langue \texttt{sentence-transformers} ;
\item la similarité cosinus a été calculée entre les documents d'entrée et les phrases-clés candidates, où celles avec les scores les plus élevés sont extraites.
\end{enumerate}

\begin{figure}[!h]
    \centering
    \includegraphics[width=1\textwidth]{img/keybert.png}
    \caption[\textit{Pipeline} de la librairie \texttt{keybert}.]{\textit{Pipeline} de la librairie \texttt{keybert}\protect\footnotemark{.}}
    \label{fig:keybert}
\end{figure}

\footnotetext{Illustration reprise de \url{https://maartengr.github.io/KeyBERT/guides/quickstart.html\#installation}.}

Une première tentative de génération des phrases-clés les plus pertinentes dans les deux corpus n'a produit que deux termes : \textsc{articulation de} [\textit{sic}] \textsc{épaule} et \textsc{paralysie faciale périphérique}. Par ailleurs, en observant les 15 phrases-clés les plus pertinentes dans le corpus \og Autres \fg{} (figure \ref{fig:keybert_autres}), nous constatons un manque de diversification des résultats et des phrases-clés qui se ressemblent (\textit{la sensibilité tactile}, \textit{sensibilité tactile au}, \textit{la sensibilité tend} etc.)\footnote{Pour assurer que les phrases-clés ne se ressemblent pas, il faut utiliser le paramètre \texttt{use\_mmr} et spécifier sa valeur entre 0 et 1.}. Un autre problème observé était la non-grammaticalité des phrases-clés extraites (\textit{sémi lunaire segment}, \textit{prière le malade} etc.), ce qui nous a incités à tester une approche plus fine, décrite dans la partie \ref{patternrank}.

\begin{figure}[!h]
    \centering
    \includegraphics[width=1\textwidth]{img/keybert_autres.png}
    \caption{Répartition des 15 termes les plus pertinents dans le corpus \og{}Autres\fg{} selon \texttt{keybert}.}
    \label{fig:keybert_autres}
\end{figure}


\subsection{Approche \textit{PatternRank}}
\label{patternrank}
Cette approche exploite la librairie \texttt{keyphrase-vectorizers} qui offre la possibilité d'extraire les phrases-clés pertinentes et spécifiques à l'aide des balises de parties de discours. Cela nous a paru comme une piste intéressante, étant donné que les termes médicaux (surtout ceux plus pointus) que l'on souhaitait extraire étaient généralement des n-grammes constitués des substantifs, suivis d'un ou plusieurs adjectifs (p. ex. \textit{sclérose latérale amyotrophique}). Voici les étapes de la chaîne de traitement de l'approche \textit{PatternRank} (figure \ref{fig:patternrank}) :
\begin{enumerate}
\item les corpus \og Charcot \fg{} et \og Autres \fg{} sont utilisés comme les données d'entrée au format \texttt{.txt} ;
\item les tokens ont été extraits et étiquetés avec les balises de partie du discours et les expressions régulières \texttt{<N.*>+<ADJ.*>*} (sans utiliser le paramètre \texttt{use\_mmr}) ;
\item les tokens ont été sélectionnés selon les balises de partie de discours souhaitées et gardés comme les phrases-clés candidates ;
\item les plongements des documents et de leurs phrases-clés candidates ont été générés par le modèle de langue (en l'occurrence \texttt{flair}\footnote{\url{https://github.com/flairNLP/flair}}) ;
\item les similarités cosinus ont été calculées entre ces deux types de plongements, et les phrases-clés candidates ont été triées par ordre décroissant ;
\item les phrases-clés les plus représentatives ont été extraites.
\end{enumerate}

\begin{figure}[!h]
    \centering
    \includegraphics[width=1\textwidth]{img/patternrank_workflow.png}
    \caption{\textit{Workflow} de la méthode \textit{PatternRank} \citep[p.~2]{schopf2022}.}
    \label{fig:patternrank}
\end{figure}

La figure \ref{fig:patternrank_partage} nous informe sur les 15 termes les plus pertinents et fréquents, extraits avec la librairie \texttt{keyphrase-vectorizers}, que l'on retrouve dans les deux corpus. Malgré certains tokens tronqués, très probablement en raison d'un \textsc{OCR} imparfait (\textit{ments} $\rightarrow$ \textit{mouvements}, \textit{decins} $\rightarrow$ \textit{médecins} etc.), nous observons une diversification des résultats. Après cela, il reste la question de mieux comprendre le rôle des phrases-clés extraites dans les écrits de l'entourage de Charcot et/ou si elles sont vraiment significatives ou pas. 

\begin{figure}[!h]
    \centering
    \includegraphics[width=1\textwidth]{img/termes_partages.png}
    \caption{Les 15 termes les plus fréquents partagés par les deux corpus selon \texttt{keyphrase-vectorizers}.}
    \label{fig:patternrank_partage}
\end{figure}


\chapter{Conclusion}
\label{conclusion}
\section{Contributions et perspectives}
%Rappel du context
Intro / Rappel Contexte

Nous avons donc pu en tirer la problématique suivante :

%Rappel des résultats

Discussion et perspectives


%Ne pas numéroter cette partie
\part*{Annexe}

%Rajouter la ligne "Annexes" dans le sommaire
%\addcontentsline{toc}{chapter}{Annexe}

\chapter{Annexe}



%changer le format des sections, subsections pour apparaittre sans le num de chapitre
\makeatletter
\renewcommand{\thesection}{\@arabic\c@section}
\makeatother

%recommencer la numérotation des section à "1"
\setcounter{section}{0}

%\section*{Liste des termes et expressions popularisées par Charcot}

%\addcontentsline{toc}{section}{\protect\numberline{}Liste des termes et expressions popularisées par Charcot}%
\begin{landscape}
\thispagestyle{empty}
\begin{table}[]
\centering
\begin{tabular}{|l|cccc|cccc|}
\hline
\textit{Terme} & Fréquence & \textsc{TF-IDF} & \textsc{BM25} & \textsc{BERT} & Fréquence & \textsc{TF-IDF} & \textsc{BM25} & \textsc{BERT} \\
\hline
% Data rows go here
\end{tabular}
\caption{Calcul de pertinence des concepts selon les métriques \textsc{TF-IDF}, \textsc{BM25} et \textsc{BERT} dans les corpus \og{}Charcot\fg{} et \og{}Autres\fg{}.}
\end{table}
\vfill
\raisebox{}{\makebox[\linewidth]{\thepage}}



\label{tab:calculs_stat}
\end{landscape}




\newpage

%récupérer les citation avec "/footnotemark"
\nocite{*}

%choix du style de la biblio
\addcontentsline{toc}{chapter}{References}
\bibliographystyle{apalike}
%inclusion de la biblio
\bibliography{bibliographie.bib}
%voir wiki pour plus d'information sur la syntaxe des entrées d'une bibliographie


\end{document}