\section{Plan de la thèse}
\hl{À METTRE À JOUR À LA FIN}
Ce mémoire est structuré en cinq parties principales : après l'introduction, nous esquissons l'évolution des théories scientifiques dans une perspective épistémologique, en prenant comme cas d'étude les contributions majeures de Charcot (chapitre \ref{rupture}).
Ensuite, nous proposons une revue de la littérature portant sur les modalités des circulations des objets patrimoniaux du point de vue numérique (chapitre \ref{sota}). Le chapitre \ref{corpus} donne un aperçu de la constitution du corpus de recherche. Le chapitre \ref{resultats} présente les premières tentatives de l'analyse computationnelle de l'impact de Charcot sur son réseau scientifique, ainsi que les limites de ces approches, en proposant une nouvelle méthode pour la quantification de la pertinence des expressions polylexicales. Enfin, le chapitre \ref{conclusion} propose une conclusion et des pistes pour des recherches futures.


